% ------------------------------------------------------------------------
% ------------------------------------------------------------------------
% Modelo de Artigo Acadêmico
% Em conformidade com:
% ABNT NBR 6022:2018: Informação e Documentação - Artigo em Publicação Periódica Científica - Apresentação
%
% Adaptado para:
% SEPA: Seminário Estudantil de Produção Acadêmica da UNIFACS
%
% Baseado na Biblioteca abnTeX2 v1.9.7
% ------------------------------------------------------------------------
% ------------------------------------------------------------------------
\documentclass[
	% Opções da classe memoir
	article,			    % indica que é um artigo acadêmico
	12pt,				    % tamanho da fonte
	oneside,			    % para impressão apenas no recto. Oposto a twoside
	a4paper,			    % tamanho do papel. 
	% Opções da classe abntex2
	chapter=TITLE,		    % títulos de capítulos convertidos em letras maiúsculas
	section=TITLE,		    % títulos de seções convertidos em letras maiúsculas
	subsection=TITLE,	    % títulos de subseções convertidos em letras maiúsculas
	%subsubsection=TITLE    % títulos de subsubseções convertidos em letras maiúsculas
	% Opções do pacote babel
	english,			    % idioma adicional para hifenização
	brazil,				    % o último idioma é o principal do documento
	sumario=tradicional
]{abntex2}
% ------------------------------------------------------------------------
% PACOTES
% ------------------------------------------------------------------------
% Pacotes fundamentais 
\usepackage{times}			    % Usa a fonte Times New Roman
\usepackage[T1]{fontenc}		% Selecao de codigos de fonte.
\usepackage[utf8]{inputenc}		% Codificacao do documento (conversão automática dos acentos)
\usepackage{indentfirst}		% Indenta o primeiro parágrafo de cada seção.
\usepackage{nomencl} 			% Lista de simbolos
\usepackage{color}				% Controle das cores
\usepackage{graphicx}			% Inclusão de gráficos
\usepackage{microtype} 			% Para melhorias de justificação
% Pacotes adicionais, usados apenas no âmbito do Modelo Canônico do abnteX2
\usepackage{lipsum}				% Para geração de dummy text
% Pacotes de citações
\usepackage[brazilian,hyperpageref]{backref}	 % Paginas com as citações na bibliografia
\usepackage[alf,bibjustif,abnt-emphasize=bf,abnt-etal-text=emph]{abntex2cite}  % Citações padrão ABNT, forçar a justificação da bibliografia e enfatizar com negrito
% Pacotes extras 
\usepackage{fancyhdr}           % Personalização do cabeçalho e rodapé
% ------------------------------------------------------------------------
% CONFIGURAÇÃO DOS PACOTES
% ------------------------------------------------------------------------
% Configurações do pacote backref
% Usado sem a opção hyperpageref de backref
\renewcommand{\backrefpagesname}{Citado na(s) página(s):~}
% Texto padrão antes do número das páginas
\renewcommand{\backref}{}
% Define os textos da citação
\renewcommand*{\backrefalt}[4]{
	\ifcase #1
		Nenhuma citação no texto.
	\or
		Citado na página #2.
	\else
		Citado #1 vezes nas páginas #2.
	\fi
}
% Configuração dos nomes padrões do babel
%\addto\captionsbrazil{
%    \renewcommand{\bibname}{Referências Bibliográficas}
%}
% Configuração do título das referências (anteriormente modificado)
\renewcommand{\bibsection}{%
    \section*{\bibname}
    \bibmark
    \ifnobibintoc\else
        \phantomsection
    \fi
    \prebibhook
}
% Modificando o tamanho da fonte "large" que é 14.4pt, para 14pt
%\renewcommand{\large}{\fontsize{14}{14}\selectfont}
% ------------------------------------------------------------------------
% DADOS DO DOCUMENTO
% ------------------------------------------------------------------------
% Informações de dados para capa
\autor{\normalsize{\textbf{Felipe Rios da Silva Cordeiro}}
\thanks{Graduando em Engenharia da Computação, UNIFACS. E-mail: felipe.rios.silva@outloook.com}}
\instituicao{
    UNIFACS - UNIVERSIDADE SALVADOR
    ESCOLA DE ARQUITETURA, ENGENHARIA\\ E TECNOLOGIA DA INFORMAÇÃO
    BACHARELADO EM ENGENHARIA DA COMPUTAÇÃO
}
\titulo{\uppercase{\normalsize{\textbf{EXECUÇÃO ESPECULATIVA:\\
LIMITES DA EXPLORAÇÃO DE INFORMAÇÕES SENSÍVEIS}}}}
\local{Salvador}
%\data{2019}
\tipotrabalho{Trabalho de Conclusão de Curso, Graduação}
\preambulo{Trabalho de conclusão de curso apresentado ao curso de graduação em Engenharia da Computação da Universidade Salvador - UNIFACS, como requisito fundamental para obtenção do título de Engenheiro da Computação.}
\orientador[Orientador:]{\normalsize{\textbf{Éldman de Oliveira Nunes}}}
%\tituloestrangeiro{Canonical article template in \abnTeX: optional foreign title}
% ------------------------------------------------------------------------
% META DADOS DO PDF
% ------------------------------------------------------------------------
% Alterando o aspecto da cor azul
\definecolor{blue}{RGB}{41,5,195}
% Informações do PDF
\makeatletter
\hypersetup{
 	%pagebackref=true,
	pdftitle={\@title}, 
	pdfauthor={\@author},
	pdfsubject={\imprimirpreambulo},
    pdfcreator={LaTeX with abnTeX2 and Overleaf},
	pdfkeywords={abnt}{latex}{abntex}{abntex2}{atigo científico}, 
	colorlinks=false,       % false: boxed links; true: colored links
	linkcolor=blue,         % color of internal links
	citecolor=blue,        	% color of links to bibliography
	filecolor=magenta,      % color of file links
	urlcolor=blue,
	bookmarksdepth=4
}
\makeatother
% ------------------------------------------------------------------------
% CONFIGURAÇÕES DAS FOLHAS E AJUSTES NAS FONTES GERAIS
% ------------------------------------------------------------------------
% Compila o índice
\makeindex
% Altera as margens
\setlrmarginsandblock{3cm}{2cm}{*}
\setulmarginsandblock{3cm}{2cm}{*}
\checkandfixthelayout
% Espaçamentos entre linhas e parágrafos 
% O tamanho do parágrafo é dado por (espaçamento na primeira linha):
\setlength{\parindent}{1.25cm}
% O espeçamento padrão é definido como \OnehalfSpacing, ou seja, um espaço e meio conforme estabelece a ABNT NBR 14724:2011
% ------------------------------------------------------------------------
% CABEÇALHOS E RODAPÉS
% ------------------------------------------------------------------------
% Criar um novo estilo de cabeçalhos e rodapés
\pagestyle{fancy}
%\setlength{\headheight}{80pt}
\fancyhf{}
%\lhead{\includegraphics[width=0.4\textwidth]{brasao.png}}
%\rhead{
%{\fontsize{8}{1.5}\selectfont
%\begin{vplace}
%TCC - TRABALHO DE CONCLUSÃO DE CURSO\break
%COORDENAÇÂO DE ENGENHARIA DA COMPUTAÇÃO\end{vplace}}}
\fancypagestyle{plain}{
    \renewcommand{\headrulewidth}{0pt}
    \renewcommand{\footrulewidth}{0pt}
    \fancyhfoffset[LE]{0mm}
    \fancyhfoffset[RE]{0mm}
    \fancyhfoffset[LO]{0mm}
    \fancyhfoffset[RO]{0mm}
}
% ------------------------------------------------------------------------
% CAPÍTULOS, SEÇÕES E SUBSEÇÕES
% ------------------------------------------------------------------------
% Chapter 12pt + Bold
\renewcommand{\ABNTEXchapterfont}{\bfseries}
\renewcommand{\ABNTEXchapterfontsize}{\normalsize}
% Section 12pt + Bold
\renewcommand{\ABNTEXsectionfont}{\bfseries}
\renewcommand{\ABNTEXsectionfontsize}{\normalsize}
% SubSection 12pt
\renewcommand{\ABNTEXsubsectionfont}{\normalfont}
\renewcommand{\ABNTEXsubsectionfontsize}{\normalsize}
% SubSubSection 12pt + Bold + Underline
\renewcommand{\ABNTEXsubsubsectionfont}{\bfseries}
\renewcommand{\ABNTEXsubsubsectionfontsize}{\normalsize}
\setsubsubsecheadstyle{\ABNTEXsubsubsectionfont\ABNTEXsubsubsectionfontsize\ABNTEXsubsubsectionupperifneeded\coloruline[black]}
% SubSubSubSection 12pt + Lowercase
\setparaheadstyle{\normalfont\ABNTEXsubsubsectionfont\ABNTEXsubsubsectionfontsize}
% Retirando espaçamentos antes dos capítulos
\setlength{\beforechapskip}{\baselineskip}
% Retirando espaçamentos depois dos capítulos
\setlength{\afterchapskip}{\baselineskip}
% Recriando a variável que instancia o resumo
\renewenvironment{resumoumacoluna}{}

% ------------------------------------------------------------------------
% INÍCIO DO DOCUMENTO
% ------------------------------------------------------------------------
\begin{document}
% Seleciona o idioma do documento (conforme pacotes do babel)
%\selectlanguage{english}
\selectlanguage{brazil}
% Retira espaço extra obsoleto entre as frases.
\frenchspacing 
% ------------------------------------------------------------------------
% ELEMENTOS PRÉ-TEXTUAIS
% ------------------------------------------------------------------------
\pretextual
\pagestyle{fancy}
% página de titulo principal (obrigatório)
%\maketitle
\begin{SingleSpace}
    \begin{center}
        \imprimirtitulo
    \end{center}
    \begin{flushright}
        \imprimirautor
        \footnote{Graduando em Engenharia da Computação, UNIFACS. E-mail: felipe.rios.silva@outloook.com}
        \\
        \imprimirorientador
        \footnote{Docente Orientador Doutor em Processamento Digital de Imagens, UNIFACS. E-mail: eldman.nunes@unifacs.br}
    \end{flushright}
\end{SingleSpace}
% Titulo em outro idioma (opcional)
% Resumo em Português
\begin{resumoumacoluna}
    \footnotesize{\begin{SingleSpace}
        \noindent
        \textbf\resumoname\\
        Conforme a ABNT NBR 6022:2018, o resumo no idioma do documento é elemento obrigatório. Constituído de uma sequência de frases concisas e objetivas e não de uma simples enumeração de tópicos, não ultrapassando 250 palavras, seguido, logo abaixo, das palavras representativas do conteúdo do trabalho, isto é, palavras-chave e/ou descritores, conforme a NBR 6028. (\ldots) As palavras-chave devem figurar logo abaixo do resumo, antecedidas da expressão Palavras-chave:, separadas entre si por ponto e finalizadas também por ponto.\\\\
        \textbf{Palavras-chave:} latex. abntex. editoração de texto.
        \vspace{\onelineskip}
    \end{SingleSpace}}
\end{resumoumacoluna}
% Resumo em Inglês
\renewcommand{\resumoname}{Abstract}
\begin{resumoumacoluna}
    \footnotesize{\begin{SingleSpace}
        \begin{otherlanguage*}{english}
            \noindent
            \textbf\resumoname\\
            Conforme a ABNT NBR 6022:2018, o resumo no idioma do documento é elemento obrigatório. Constituído de uma sequência de frases concisas e objetivas e não de uma simples enumeração de tópicos, não ultrapassando 250 palavras, seguido, logo abaixo, das palavras representativas do conteúdo do trabalho, isto é, palavras-chave e/ou descritores, conforme a NBR 6028. (\ldots) As palavras-chave devem figurar logo abaixo do resumo, antecedidas da expressão Palavras-chave:, separadas entre si por ponto e finalizadas também por ponto.\\\\
            \textbf{Keywords:} latex. abntex.
        \end{otherlanguage*}
    \end{SingleSpace}}
\end{resumoumacoluna}
% ------------------------------------------------------------------------
% ELEMENTOS TEXTUAIS
% ------------------------------------------------------------------------
\textual
\pagestyle{fancy}
% ------------------------------------------------------------------------
% INTRODUÇÃO
% ------------------------------------------------------------------------
\section{Introdução}
% ---------------------------------
% TEMA
% ---------------------------------
A execução especulativa é uma técnica de projeto de microarquitetura, que proporciona o aprimoramento da velocidade de processamento nas CPUs modernas. Esta técnica consiste na estimativa e execução de instruções com valores ainda não conhecidos pela CPU, durante um período curto de inatividade. Do ponto de vista funcional, esta especulação traria problemas se os resultados de especulações incorretas fossem efetivados. Porém, quando a verdadeira informação é recuperada, a CPU verifica a exatidão da suposição e descarta o ``caminho'' (fluxo de execução) que foi executado incorretamente, eliminando os valores nos registradores ou alterações em variáveis, por exemplo.

Utilizando ataques de canais laterais, \citeonline{Lipp2018meltdown} e \citeonline{Kocher2018spectre} provaram que é possível recuperar informações privilegiadas que foram especuladas, através da memória cache, que não é revertida quando uma especulação errônea da CPU acontece. A partir disto, com o conhecimento microarquitetural necessário, é possível um atacante induzir a execução especulativa errônea, por ``viciar'' o processador em uma cadeia de especulações e transferir as informações especuladas para um canal lateral (alternativo) que pode ser lido posteriormente.

\begin{comment}
\subsection{Tema}
A execução especulativa é uma técnica de projeto de microarquitetura, que proporciona o aprimoramento da velocidade de processamento nos processadores modernos. Está presente em muitos processadores de vários fabricantes, incluindo \emph{Intel}, \emph{AMD} e \emph{ARM Holdings}. Esta técnica consiste na estimativa e execução de instruções, com valores ainda não conhecidos pela CPU, durante um período curto de inatividade (que acontece durante a espera de valores reais, provenientes da memória principal, que é mais lenta do que a memória cache).

Do ponto de vista de funcional, esta especulação traria problemas se os resultados de especulações incorretas fossem efetivados. Porém, quando a verdadeira informação é recuperada, a CPU verifica a exatidão da suposição e descarta o ``caminho'' (fluxo de execução) que foi executado incorretamente. Eliminando valores nos registradores, ou alterações em variáveis por exemplo.

Apoderando-se do conhecimento microarquitetural necessário para se conhecer em quais situações e em quais instruções a execução especulativa ocorre, é possível um atacante forçar ou induzir a execução especulativa, por ``viciar'' o processador em uma cadeia de especulações e transferir as informações especuladas para um canal alternativo (como por exemplo, a memória cache). Caso a transferência seja bem-sucedida, o atacante efetua a leitura dos dados presentes no canal alternativo, neste caso na memória cache, medindo o tempo de acesso aos dados que foram especulados e comparando-os com um tempo médio de acesso (à memória cache) conhecido. Caso a informação demore de ser recuperada (baseando-se na média), supõe-se que ela não se encontra na memória cache. Caso contrário, o atacante acertou a suposição da informação correta.
\end{comment}
% ---------------------------------
% JUSTIFICATIVA
% ---------------------------------
Os ataques de canais laterais dirigidos a \emph{softwares} cujas estruturas são conhecidas, acontecem em condições específicas. Quando por exemplo, existem regiões compartilhadas de memória entre a vítima e o atacante. Mesmo assim, a abrangência destes ataques é considerável: é possível afirmar que chips da \emph{Intel} desde 1995 (exceto \emph{Intel Itanium} e \emph{Intel Atom} antes de 2013), alguns de arquitetura \emph{ARM} (\emph{Advanced RISC Machine}) e outros feitos pela \emph{AMD}, foram afetados. Servidores na nuvem, celulares, desktops, notebooks e basicamente todos os chips que podem manter muitas instruções em execução, podem ter dados sigilosos comprometidos. Segundo o \emph{Microcode Revision Guidance} da \emph{Intel}, 193 MCUs (\emph{Microcontroller Unit}) irão continuar com a falha, pois apresentaram instabilidade com a mitigação das vulnerabilidades \cite{intel-mug}.

\begin{comment}
\subsection{Justificativa}
Três grupos compostos por: Jann Horn (\emph{Google Project Zero}); Werner Haas e Thomas Prescher (\emph{Cyberus Technology}); Daniel Gruss, Moritz Lipp, Stefan Mangard e Michael Schwarz (\emph{Graz University of Technology}), em descobertas independentes relataram duas vulnerabilidades na arquitetura dos processadores de design moderno das fabricantes: \emph{Intel}, \emph{AMD} e \emph{ARM Holdings} (comunicado as fabricantes em 1º de Junho de 2017 e divulgado ao público em 03 de Janeiro de 2018). Essas duas falhas de segurança batizadas de \emph{Spectre}\footnote{Especulativo, espectro, ou fantasma (``speculative'', inglês). O nome deriva da causa da falha.} \cite{Kocher2018spectre} e \emph{Meltdown}\footnote{Colapso, ou derreter (``melt'', inglês). O nome deriva da consequência da exploração da falha.} \cite{Lipp2018meltdown} chamam a atenção pois, segundo o estudo detalhado feito pelos grupos, elas se aproveitam da execução especulativa das MCUs (\emph{Microcontroller Unit}).

Em termos quantitativos, chips da \emph{Intel} desde de 1995 (exceto \emph{Intel Itanium} e \emph{Intel Atom} antes de 2013), alguns de arquitetura \emph{ARM} (\emph{Advanced RISC Machine}) e outros feitos pela \emph{AMD}, foram afetados. Servidores na nuvem, celulares, desktops, notebooks e basicamente todos os chips que podem manter muitas instruções em execução, podem ter dados sigilosos comprometidos. E, segundo o \emph{Microcode Revision Guidance} da \emph{Intel}, 193 MCUs irão continuar com a falha, pois apresentaram questões de instabilidade com a mitigação das vulnerabilidades \cite{intel-mug}.

Em uma nota oficial a \emph{Intel} se pronunciou evidenciando seu comprometimento em mitigar as falhas, estudando e disponibilizando \emph{firmwares} para corrigi-las (ou atenuá-las). Se defendendo também de especulações em relação ao desempenho dos processadores, depois da correção via \emph{software}, afirmou que qualquer diferença no desempenho ``depende da carga de trabalho e, para o usuário médio do computador, não deve ser significativo e será atenuado com o tempo'' \cite{intel-news-001}. Depois disso, publicou um informativo de segurança \cite{intel-sa-00088}, informando a lista de processadores da marca atingidos pelas falhas e suas ramificações.

As outras fabricantes (\emph{AMD} e \emph{ARM Holdings}), em parceria com algumas montadoras e produtoras de \emph{softwares} também se pronunciaram em notas oficiais, assumindo ou não as falhas em seus produtos e, tomando certa medida de prevenção, formaram parcerias para lançarem correções em aplicações para usuário final (\emph{browsers} por exemplo), que previnem a exploração das falhas. Fabricantes de jogos que utilizam os processadores \emph{AMD} ou \emph{ARM}, como \emph{Sony} e \emph{Nintendo}, não se pronunciaram, conforme lista oficial publicada pela Universidade de Tecnologia de Graz \cite{meltdownspectreattack}.

Em 03 de Janeiro de 2018, pesquisadores da Universidade Católica de Leuven e do IT de Israel, das Universidades de Michigan e Adelaide e da \emph{CSIRO Data61}, fizeram novas publicações comprovativas de duas novas derivações da \emph{Spectre}. Batizadas de \emph{Foreshadow}\footnote{Prefigurar, pressupor ou prévio (``fore'', inglês). Assim como no \footnotemark[\numexpr\value{footnote}-2], o nome deriva da causa da falha.} e \emph{Foreshadow-NG}, em processadores de servidores e em máquinas virtuais \cite{vanbulck2018foreshadow}.

Em 01 de Março de 2019, pesquisadores do Instituto Politécnico de Worcester e da Universidade de Lübeck, publicaram um artigo expondo mais uma possibilidade de exploração do comportamento arquitetural dos processadores da \emph{Intel}. Chamada de \emph{Spoiler}, esta nova vulnerabilidade não é uma variação das anteriores e nenhum dos métodos de mitigação divulgados até a data desta pesquisa, suprimem esta nova forma de realizar vazamentos de informações \cite{islam2019spoiler}.

Tal assunto adquiriu grande importância na divulgação pelos veículos de comunicação, pois é provável que muitas máquinas, aparelhos móveis, \emph{datacenters} e consequentemente informações sigilosas sejam expostas e continuem sendo. Isto se dá, porque tal categoria de vulnerabilidade está em nível de \emph{hardware}, e correções via \emph{software} são paliativos que custam questões de desempenho.
\end{comment}
% ---------------------------------
% PROBLEMA DE PESQUISA
% ---------------------------------
Os ataques de canais laterais que se aproveitam da execução especulativa, constituem-se um assunto relevante devido a probabilidade de que muitas máquinas, aparelhos móveis, \emph{datacenters} e consequentemente informações sigilosas estejam expostas. Isto acontece porque, tal categoria de vulnerabilidade se encontra em nível de \emph{hardware}, e correções via \emph{software} são paliativos que custam questões de desempenho e estabilidade. 

Considerando que os autores citados utilizaram \emph{softwares} com estruturas conhecidas (endereços de memória e fluxos condicionais) em seus experimentos, algumas questões ainda permanecem em aberto. Desse modo, é possível explorar informações sensíveis de \emph{softwares} cujo fluxo condicional e estrutura de memória sejam desconhecidos por meio da exploração da execução especulativa induzida de forma errônea?

\begin{comment}
\subsection{Problema de Pesquisa}
\citeonline{Kocher2018spectre} em seu artigo, não definiu bem algumas questões sobre a expansão dos ataques que se aproveitam da execução especulativa, como por exemplo se: há ou não possibilidades de vazar informações de softwares cujo fluxo o atacante não conhece; há ou não possibilidade de utilizar endereços de memória desconhecidos inicialmente, para recuperar informações de outros softwares em execução; existe possibilidade de fazer tais ataques em softwares de terceiros.

Isto porquê, quando se descreveu os ataques utilizando a execução especulativa, os experimentos e testes demonstrados foram feitos em softwares criados pelos próprios descobridores da vulnerabilidade. Ou seja, eles possuíam endereços de memória e fluxos condicionais bem definidos para garimpar as informações sigilosas. Diante disto, é proposto uma pesquisa para responder: é possível explorar informações sensíveis de softwares cujo fluxo condicional e estrutura de memória sejam desconhecidos por meio do emprego da execução especulativa?
\end{comment}
% ---------------------------------
% OBJETIVOS
% ---------------------------------
Em razão disto, esta pesquisa visa esclarecer, explicar e aplicar métodos desenvolvidos pelos pesquisadores supramencionados, para expor informações de forma genérica em \emph{softwares} que não compartilham da mesma estrutura. Ou seja, que não compartilham a mesma região de memória virtual, e que ocupam regiões dentro do espaço disponível para o usuário. Demonstrando assim, como a exploração da execução especulativa pode ocorrer em \emph{softwares} cujo o fluxo condicional e estrutura de memória se desconhece, afim de ``vazar'' informações sensíveis.

\begin{comment}
\subsection{Objetivos}
Em razão dos fatos supramencionados, esta pesquisa visa esclarecer, explicar e aplicar métodos desenvolvidos pelos autores citados anteriormente para expor informações de forma genérica (em \emph{softwares} comuns) em computadores pessoais. Assim, o \textbf{objetivo geral} desta pesquisa é demonstrar como a execução especulativa pode ocorrer em \emph{softwares} cujo o fluxo condicional e estrutura de memória se desconhece.

Para alcançar este \textbf{objetivo geral}, os seguintes \textbf{objetivos específicos} foram levantados: 
\begin{enumerate}
    \item \label{o1} Demonstrar e explorar a falha de execução especulativa em fluxos condicionais conhecidos;
    \item \label{o2} Demonstrar e explorar a falha de execução especulativa em fluxos condicionais desconhecidos;
    \item \label{o3} Demonstrar e explorar a falha de execução especulativa em regiões de memória cujos endereços não são conhecidos inicialmente.
\end{enumerate}
\end{comment}
% ---------------------------------
% QUESTÕES DE PESQUISA
% ---------------------------------
\begin{comment}
\subsection{Questões de Pesquisa}
\begin{itemize}
    \item Questões abordadas por esta pesquisa:
    \begin{enumerate}
	    \item \label{q1} Quais informações, como fluxo condicional e estrutura de memória, são necessárias para se aplicar execução especulativa em um software?
	    \item \label{q2} Além de usar o tempo de retorno da memória cache para extrair as informações da própria memória cache, quais ataques de hardware também podem ser usados para desviar dados deste canal lateral?
	    \item \label{q3} Quais endereços da memória cache são acessíveis a um programa de nível de usuário, que podem serem utilizados como canal alternativo?
	    \item \label{q4} Quais são os outros canais laterais existentes (além da memória cache), que outros autores já exploraram?
	\end{enumerate}
    \item Questões de pesquisa para trabalhos futuros:
    \begin{enumerate}
	    \item \label{qf1} Como executar instruções mais complexas, além de somente desvios de informação, aproveitando-se da execução especulativa?
	    \item \label{qf1} Segundo \citeonline{Kocher2018spectre} até mesmo códigos que não contenham instruções com ramificações condicionais estão em risco. Como explorá-los?
	\end{enumerate}
\end{itemize}
\end{comment}
% ---------------------------------
% PROCEDIMENTOS METODOLÓGICOS
% ---------------------------------
Para a obtenção de êxito nesta busca por limites de exploração destes dados sensíveis, esta pesquisa teve seus procedimentos e resultados técnicos analisados e avaliados em máquinas virtuais e reais, induzidas ao ambiente que se é esperado para a realização dos testes. Portanto, os testes foram analisados e testados mais de uma vez, em ambientes diferentes conforme o desenvolvimento da exploração feita pelo autor. O intuito disto é promover o aprofundamento de um conhecimento já exposto através de pesquisas anteriores, não se descartando a possibilidade de produção de um conhecimento útil para estudos futuros sobre o assunto. Sendo assim, o autor espera complementar alguns aspectos e peculiaridades de pesquisas anteriormente feitas, preenchendo lacunas de conhecimento a respeito da execução especulativa e dos ataques de canais laterais.

% ---------------------------------
% ORGANIZAÇÃO DO ARTIGO
% ---------------------------------
Para melhor segmentação e explanação do conhecimento visando o entendimento do leitor, as outras seções do artigo estão dispostas como se segue: a seção 2, \textbf{referencial teórico}, aborda os conceitos de \emph{pipeline}, \emph{branch condition}, \emph{branch target buffer}, \emph{cache levels} e execução especulativa, além dos trabalhos relacionados a ataques de canais laterais. A seção 3, \textbf{referencial metodológico}, classifica a pesquisa e detalha os procedimentos metodológicos empregados. A seção 4, \textbf{resultados e discussões}, apresenta os testes e análises feitas a partir de códigos de outros pesquisadores e artigos relacionados, visando solucionar o problema descrito. E a seção 5, apresenta a \textbf{conclusão}, discute os resultados alcançados, limitações encontradas, perspectivas futuras e conclusões finais.

% ------------------------------------------------------------------------
% REFERENCIAL METODOLÓGICO
% ------------------------------------------------------------------------
\section{Referencial Metodológico}
Classificação da pesquisa. Apresentação, classificação e descrição dos métodos e dos procedimentos técnicos utilizados neste trabalho.
% ---------------------------------
% CLASSIFICAÇÃO DA PESQUISA
% ---------------------------------
\subsection{Método}
Esta é uma pesquisa com finalidade de natureza básica, com objetivos de caráter descritivos, que utiliza uma abordagem quantitativa, com procedimentos fundamentados em pesquisa bibliográfica, documental e experimental. Caracterizada por um estudo transversal, levando em conta os resultados de análises e testes conduzidos em um ambiente previamente configurado, ou seja, em um laboratório.

Trata-se de uma \textbf{pesquisa básica} porque além de promover o aprofundamento de um conhecimento já exposto através de pesquisas anteriores, não se descarta a possibilidade de produção de um conhecimento útil para estudos futuros sobre o assunto. Sendo assim, o autor espera complementar alguns aspectos e peculiaridades de pesquisas anteriormente feitas, preenchendo lacunas de conhecimento a respeito da execução especulativa.

É uma \textbf{pesquisa descritiva}, pois tem por objetivo demonstrar a possibilidade de execução da falha de especulação em circunstâncias específicas e diferentes das circunstâncias abordadas no artigo de \citeonline{Kocher2018spectre}. Essa comparação leva em conta artigos de outros autores que tentaram defender o processador contra o ataque utilizando outros métodos via \emph{software}. Desta forma, existe uma associação entre a proteção para determinadas situações e o ataque, com determinadas circunstâncias (endereços de memória, \emph{softwares} a serem explorados e técnicas para isto).

A pesquisa utiliza uma \textbf{abordagem quantitativa}, visto que a avaliação dos resultados possui caráter bem definido e exato do ponto de vista funcional. Avaliando a execução e não execução dos procedimentos que levam a falha e consequentemente ao vazamento, é possível concluir, de forma objetiva e direta, se a técnica funciona ou não. Tais resultados foram comprovados através de testes.

Os procedimentos técnicos adotados pelo autor baseiam-se em técnicas de pequisa bibliográfica, documental e experimental. O \textbf{procedimento bibliográfico} foi adotado pois proporciona a condição de comparação histórica com a literatura, além de conferir o embasamento teórico veraz necessário (desconsiderando literaturas sensacionalistas, imprecisas ou sem fontes sólidas), creditando aos autores primários suas devidas contribuições para o ponto atingido deste trabalho. O \textbf{procedimento documental} confere a garantia de consulta e comparação dos artigos primários, com a literatura não processada (manuais, guias e notas oficiais dos fabricantes). E o \textbf{procedimento experimental} proporcionou ao autor agregar valor ao trabalho em questão através de testes e experimentos. Para isto, os objetos de estudo foram os processadores físicos disponíveis, e os \emph{exploits} foram os meios de controle utilizados para modificar o comportamento dos objetos de estudo. Visando o acréscimo de todo conhecimento anterior.

Estes procedimentos técnicos, juntamente com os resultados, foram analisados e avaliados durante um determinado período de tempo. Sendo que, as variáveis observadas são relativas a momentos instantâneos de regiões dinâmicas de memória, conferindo uma característica \textbf{transversal} ao estudo, tratando-se de tempo de aplicação.

A localização física desta pesquisa é irrelevante (por ter como objetivo apurar informações lógicas), contudo foi conduzida em \textbf{máquinas virtuais e reais}. Um ambiente laboratorial previamente configurado foi utilizado para simular tanto a ausência de defesa (contra o vazamento de informações), como também a presença dos mecanismos de defesa. Por isso, foi necessário testar cada cenário criado de forma exaustiva, utilizando os \emph{exploits} em ordem cronológica, respeitando a ordem de, condições conhecidas para, condições desconhecidas.
% ---------------------------------
% CLASSIFICAÇÃO DOS PROCEDIMENTOS
% ---------------------------------
\subsection{Procedimentos Metodológicos}
Descreva os parâmetros as variáveis observadas (dependente, independente, moderadoras, de controle, extrínsecas interveniente e antecedente), os parâmetros estatísticos (população, amostra, margem de erro, confiabilidade, tipo de amostragem)
Descreva o método ou modelo o processo de desenvolvimento (cascata, espiral, prototipagem evolutiva ou outro), os modelos e artefatos usados
Descreva as etapas de investigação atividades envolvidas, a forma de levantamento de requisitos
Descreva as condições de execução o tempo, o local, as pessoas envolvidas (funcionalmente)
Descreva os dados , as técnicas de coleta de dados (BD, entrevistas, questionário, formulário, outro), a técnica de análise de dados (Quantitativa – análise estatística; Qualitativa – análise de conteúdo)
Descreva os recursos, os materiais, os equipamentos e os programas envolvidos
Outros: descreva os procedimentos éticos se a pesquisa for envolver seres humanos

Exemplo:

Para este estudo, foram coletados dados a partir de uma base de registros de 93 pacientes da UTI do Hospital São Vicente de Paulo, na região norte do Estado do Rio Grande do Sul em 2011, e foram cedidos por um membro da equipe que realizou o trabalho. Foram obtidos dados ventilatórios e fisiológicos dos pacientes, no momento da admissão, e em especial pacientes adultos e com pneumonia. O estudo foi aprovado pelo Comitê de Ética da Universidade de Passo Fundo, sob o parecer 453/2011, sem a necessidade de consentimento informado do paciente, Sachetti (2014).

Os recursos empregados neste trabalho foram: o ambiente MATLAB®, versão 9.4.0 R2018a, o simulador de ventilação mecânica DIXTAL 3010, computador com processador Intel® Core™ 2 Duo CPU T9400 de 2,53 GHz com 4,00 GB de memória RAM DDR2, 120 GB de disco e sistema operacional de 64 Bits.

% Finaliza a parte no bookmark do PDF, para que se inicie o bookmark na raiz
\bookmarksetup{startatroot}
% ------------------------------------------------------------------------
% CONCLUSÃO
% ------------------------------------------------------------------------
%\section{Considerações finais}
%\lipsum[31]
% ------------------------------------------------------------------------
% ELEMENTOS PÓS-TEXTUAIS
% ------------------------------------------------------------------------
\postextual
% ------------------------------------------------------------------------
% REFERẼNCIAS
% ------------------------------------------------------------------------
\bibliography{referencias}
% ------------------------------------------------------------------------
% APÊNDICES
% ------------------------------------------------------------------------
\begin{apendicesenv}
%\chapter{Cras non urna sed feugiat cum sociis natoque penatibus et magnis dis parturient montes nascetur ridiculus mus}
%\lipsum[31]
\end{apendicesenv}
% ------------------------------------------------------------------------
% ANEXOS
% ------------------------------------------------------------------------
\begin{anexosenv}
\vspace{\onelineskip}
%\chapter{Cras non urna sed feugiat cum sociis natoque penatibus et magnis dis parturient montes nascetur ridiculus mus}
%\lipsum[31]
\end{anexosenv}
% ------------------------------------------------------------------------
% AGRADECIMENTOS
% ------------------------------------------------------------------------
\section*{Agradecimentos}
%\lipsum[31]
% ------------------------------------------------------------------------
% FINAL DO DOCUMENTO
% ------------------------------------------------------------------------
\end{document}