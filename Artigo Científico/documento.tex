% ------------------------------------------------------------------------
% ------------------------------------------------------------------------
% Modelo de Artigo Acadêmico
% Em conformidade com:
% ABNT NBR 6022:2018: Informação e Documentação - Artigo em Publicação Periódica Científica - Apresentação
%
% Adaptado para:
% SEPA: Seminário Estudantil de Produção Acadêmica da UNIFACS
%
% Baseado na Biblioteca abnTeX2 v1.9.7
% ------------------------------------------------------------------------
% ------------------------------------------------------------------------
\documentclass[
	% Opções da classe memoir
	article,			    % indica que é um artigo acadêmico
	12pt,				    % tamanho da fonte
	oneside,			    % para impressão apenas no recto. Oposto a twoside
	a4paper,			    % tamanho do papel. 
	% Opções da classe abntex2
	chapter=TITLE,		    % títulos de capítulos convertidos em letras maiúsculas
	section=TITLE,		    % títulos de seções convertidos em letras maiúsculas
	subsection=TITLE,	    % títulos de subseções convertidos em letras maiúsculas
	%subsubsection=TITLE    % títulos de subsubseções convertidos em letras maiúsculas
	% Opções do pacote babel
	english,			    % idioma adicional para hifenização
	brazil,				    % o último idioma é o principal do documento
	sumario=tradicional
]{abntex2}
% ------------------------------------------------------------------------
% PACOTES
% ------------------------------------------------------------------------
% Pacotes fundamentais 
\usepackage{times}			    % Usa a fonte Times New Roman
\usepackage[T1]{fontenc}		% Selecao de codigos de fonte.
\usepackage[utf8]{inputenc}		% Codificacao do documento (conversão automática dos acentos)
\usepackage{indentfirst}		% Indenta o primeiro parágrafo de cada seção.
\usepackage{nomencl} 			% Lista de simbolos
\usepackage{color}				% Controle das cores
\usepackage{graphicx}			% Inclusão de gráficos
\usepackage{microtype} 			% Para melhorias de justificação
% Pacotes adicionais, usados apenas no âmbito do Modelo Canônico do abnteX2
\usepackage{lipsum}				% Para geração de dummy text
% Pacotes de citações
\usepackage[brazilian,hyperpageref]{backref}	 % Paginas com as citações na bibliografia
\usepackage[alf,bibjustif,abnt-emphasize=bf]{abntex2cite}  % Citações padrão ABNT, forçar a justificação da bibliografia e enfatizar com negrito
% Pacotes extras 
\usepackage{fancyhdr}           % Personalização do cabeçalho e rodapé
% ------------------------------------------------------------------------
% CONFIGURAÇÃO DOS PACOTES
% ------------------------------------------------------------------------
% Configurações do pacote backref
% Usado sem a opção hyperpageref de backref
\renewcommand{\backrefpagesname}{Citado na(s) página(s):~}
% Texto padrão antes do número das páginas
\renewcommand{\backref}{}
% Define os textos da citação
\renewcommand*{\backrefalt}[4]{
	\ifcase #1
		Nenhuma citação no texto.
	\or
		Citado na página #2.
	\else
		Citado #1 vezes nas páginas #2.
	\fi
}
% Configuração dos nomes padrões do babel
%\addto\captionsbrazil{
%    \renewcommand{\bibname}{Referências Bibliográficas}
%}
% Configuração do título das referências (anteriormente modificado)
\renewcommand{\bibsection}{%
    \section*{\bibname}
    \bibmark
    \ifnobibintoc\else
        \phantomsection
    \fi
    \prebibhook
}
% Modificando o tamanho da fonte "large" que é 14.4pt, para 14pt
%\renewcommand{\large}{\fontsize{14}{14}\selectfont}
% ------------------------------------------------------------------------
% DADOS DO DOCUMENTO
% ------------------------------------------------------------------------
% Informações de dados para capa
\autor{\normalsize{\textbf{Felipe Rios da Silva Cordeiro}}
\thanks{Graduando em Engenharia da Computação, UNIFACS. E-mail: felipe.rios.silva@outloook.com}}
\instituicao{
    UNIFACS - UNIVERSIDADE SALVADOR
    ESCOLA DE ARQUITETURA, ENGENHARIA\\ E TECNOLOGIA DA INFORMAÇÃO
    BACHARELADO EM ENGENHARIA DA COMPUTAÇÃO
}
\titulo{\uppercase{\normalsize{\textbf{EXECUÇÃO ESPECULATIVA:\\
LIMITES DA EXPLORAÇÃO DE INFORMAÇÕES SENSÍVEIS}}}}
\local{Salvador}
%\data{2019}
\tipotrabalho{Trabalho de Conclusão de Curso, Graduação}
\preambulo{Trabalho de conclusão de curso apresentado ao curso de graduação em Engenharia da Computação da Universidade Salvador - UNIFACS, como requisito fundamental para obtenção do título de Engenheiro da Computação.}
\orientador[Orientador:]{\normalsize{\textbf{Éldman de Oliveira Nunes}}}
%\tituloestrangeiro{Canonical article template in \abnTeX: optional foreign title}
% ------------------------------------------------------------------------
% META DADOS DO PDF
% ------------------------------------------------------------------------
% Alterando o aspecto da cor azul
\definecolor{blue}{RGB}{41,5,195}
% Informações do PDF
\makeatletter
\hypersetup{
 	%pagebackref=true,
	pdftitle={\@title}, 
	pdfauthor={\@author},
	pdfsubject={\imprimirpreambulo},
    pdfcreator={LaTeX with abnTeX2 and Overleaf},
	pdfkeywords={abnt}{latex}{abntex}{abntex2}{atigo científico}, 
	colorlinks=false,       % false: boxed links; true: colored links
	linkcolor=blue,         % color of internal links
	citecolor=blue,        	% color of links to bibliography
	filecolor=magenta,      % color of file links
	urlcolor=blue,
	bookmarksdepth=4
}
\makeatother
% ------------------------------------------------------------------------
% CONFIGURAÇÕES DAS FOLHAS E AJUSTES NAS FONTES GERAIS
% ------------------------------------------------------------------------
% Compila o índice
\makeindex
% Altera as margens
\setlrmarginsandblock{3cm}{2cm}{*}
\setulmarginsandblock{3cm}{2cm}{*}
\checkandfixthelayout
% Espaçamentos entre linhas e parágrafos 
% O tamanho do parágrafo é dado por (espaçamento na primeira linha):
\setlength{\parindent}{1.25cm}
% O espeçamento padrão é definido como \OnehalfSpacing, ou seja, um espaço e meio conforme estabelece a ABNT NBR 14724:2011
% ------------------------------------------------------------------------
% CABEÇALHOS E RODAPÉS
% ------------------------------------------------------------------------
% Criar um novo estilo de cabeçalhos e rodapés
\pagestyle{fancy}
%\setlength{\headheight}{80pt}
\fancyhf{}
%\lhead{\includegraphics[width=0.4\textwidth]{brasao.png}}
%\rhead{
%{\fontsize{8}{1.5}\selectfont
%\begin{vplace}
%TCC - TRABALHO DE CONCLUSÃO DE CURSO\break
%COORDENAÇÂO DE ENGENHARIA DA COMPUTAÇÃO\end{vplace}}}
\fancypagestyle{plain}{
    \renewcommand{\headrulewidth}{0pt}
    \renewcommand{\footrulewidth}{0pt}
    \fancyhfoffset[LE]{0mm}
    \fancyhfoffset[RE]{0mm}
    \fancyhfoffset[LO]{0mm}
    \fancyhfoffset[RO]{0mm}
}
% ------------------------------------------------------------------------
% CAPÍTULOS, SEÇÕES E SUBSEÇÕES
% ------------------------------------------------------------------------
% Chapter 12pt + Bold
\renewcommand{\ABNTEXchapterfont}{\bfseries}
\renewcommand{\ABNTEXchapterfontsize}{\normalsize}
% Section 12pt + Bold
\renewcommand{\ABNTEXsectionfont}{\bfseries}
\renewcommand{\ABNTEXsectionfontsize}{\normalsize}
% SubSection 12pt
\renewcommand{\ABNTEXsubsectionfont}{\normalfont}
\renewcommand{\ABNTEXsubsectionfontsize}{\normalsize}
% SubSubSection 12pt + Bold + Underline
\renewcommand{\ABNTEXsubsubsectionfont}{\bfseries}
\renewcommand{\ABNTEXsubsubsectionfontsize}{\normalsize}
\setsubsubsecheadstyle{\ABNTEXsubsubsectionfont\ABNTEXsubsubsectionfontsize\ABNTEXsubsubsectionupperifneeded\coloruline[black]}
% SubSubSubSection 12pt + Lowercase
\setparaheadstyle{\normalfont\ABNTEXsubsubsectionfont\ABNTEXsubsubsectionfontsize}
% Retirando espaçamentos antes dos capítulos
\setlength{\beforechapskip}{\baselineskip}
% Retirando espaçamentos depois dos capítulos
\setlength{\afterchapskip}{\baselineskip}
% Recriando a variável que instancia o resumo
\renewenvironment{resumoumacoluna}{}

% ------------------------------------------------------------------------
% INÍCIO DO DOCUMENTO
% ------------------------------------------------------------------------
\begin{document}
% Seleciona o idioma do documento (conforme pacotes do babel)
%\selectlanguage{english}
\selectlanguage{brazil}
% Retira espaço extra obsoleto entre as frases.
\frenchspacing 
% ------------------------------------------------------------------------
% ELEMENTOS PRÉ-TEXTUAIS
% ------------------------------------------------------------------------
\pretextual
\pagestyle{fancy}
% página de titulo principal (obrigatório)
%\maketitle
\begin{SingleSpace}
    \begin{center}
        \imprimirtitulo
    \end{center}
    \begin{flushright}
        \imprimirautor
        \footnote{Graduando em Engenharia da Computação, UNIFACS. E-mail: felipe.rios.silva@outloook.com}
        \\
        \imprimirorientador
        \footnote{Docente Orientador Doutor em Processamento Digital de Imagens, UNIFACS. E-mail: eldman.nunes@unifacs.br}
    \end{flushright}
\end{SingleSpace}
% Titulo em outro idioma (opcional)
% Resumo em Português
\begin{resumoumacoluna}
    \footnotesize{\begin{SingleSpace}
        \noindent
        \textbf\resumoname\\
        Conforme a ABNT NBR 6022:2018, o resumo no idioma do documento é elemento obrigatório. Constituído de uma sequência de frases concisas e objetivas e não de uma simples enumeração de tópicos, não ultrapassando 250 palavras, seguido, logo abaixo, das palavras representativas do conteúdo do trabalho, isto é, palavras-chave e/ou descritores, conforme a NBR 6028. (\ldots) As palavras-chave devem figurar logo abaixo do resumo, antecedidas da expressão Palavras-chave:, separadas entre si por ponto e finalizadas também por ponto.\\\\
        \textbf{Palavras-chave:} latex. abntex. editoração de texto.
        \vspace{\onelineskip}
    \end{SingleSpace}}
\end{resumoumacoluna}
% Resumo em Inglês
\renewcommand{\resumoname}{Abstract}
\begin{resumoumacoluna}
    \footnotesize{\begin{SingleSpace}
        \begin{otherlanguage*}{english}
            \noindent
            \textbf\resumoname\\
            Conforme a ABNT NBR 6022:2018, o resumo no idioma do documento é elemento obrigatório. Constituído de uma sequência de frases concisas e objetivas e não de uma simples enumeração de tópicos, não ultrapassando 250 palavras, seguido, logo abaixo, das palavras representativas do conteúdo do trabalho, isto é, palavras-chave e/ou descritores, conforme a NBR 6028. (\ldots) As palavras-chave devem figurar logo abaixo do resumo, antecedidas da expressão Palavras-chave:, separadas entre si por ponto e finalizadas também por ponto.\\\\
            \textbf{Keywords:} latex. abntex.
        \end{otherlanguage*}
    \end{SingleSpace}}
\end{resumoumacoluna}
% ------------------------------------------------------------------------
% ELEMENTOS TEXTUAIS
% ------------------------------------------------------------------------
\textual
\pagestyle{fancy}
% ------------------------------------------------------------------------
% INTRODUÇÃO
% ------------------------------------------------------------------------
\section{Introdução}

\subsection{Tema}
A execução especulativa é uma técnica de projeto de microarquitetura, que proporciona o aprimoramento da velocidade de processamento nos processadores modernos. Está presente em muitos processadores de vários fabricantes, incluindo Intel, AMD e ARM. Esta técnica consiste na estimativa e execução de instruções, com valores ainda não conhecidos pela CPU, durante um período curto de inatividade (que acontece durante a espera de valores reais, provenientes da memória principal, que é mais lenta do que a memória cache).

Do ponto de vista de funcional, esta especulação traria problemas se os resultados de especulações incorretas fossem efetivados. Porém, quando a verdadeira informação é recuperada, a CPU verifica a exatidão da suposição e descarta o “caminho” (fluxo de execução) que foi executado incorretamente. Eliminando valores nos registradores, ou alterações em variáveis por exemplo.

Apoderando-se do conhecimento micro arquitetural necessário para se conhecer em quais situações e em quais instruções a execução especulativa ocorre, é possível um atacante forçar ou induzir a execução especulativa, por "viciar" o processador em uma cadeia de especulações e transferir as informações especuladas para um canal alternativo (como por exemplo, a memória cache). Caso a transferência seja bem-sucedida, o atacante efetua a leitura dos dados presentes no canal alternativo, neste caso na memória cache, medindo o tempo de acesso aos dados que foram especulados e comparando-os com um tempo médio de acesso (à memória cache) conhecido. Caso a informação demore de ser recuperada (baseando-se na média), supõe-se que ela não se encontra na memória cache. Caso contrário, o atacante acertou a suposição da informação correta.

\subsection{Problema de Pesquisa}
Jann Horn em seu artigo, não definiu bem algumas questões sobre a expansão dos ataques que se aproveitam da execução especulativa, como por exemplo se: há ou não possibilidades de vazar informações de softwares cujo fluxo o atacante não conhece; há ou não possibilidade de utilizar endereços de memória desconhecidos incialmente, para recuperar informações de outros softwares em execução; existe possibilidade de fazer tais ataques em softwares de terceiros.

Isto porquê, quando se descreveu os ataques utilizando a execução especulativa, os experimentos e testes demonstrados foram feitos em softwares criados pelos próprios descobridores da vulnerabilidade. Ou seja, eles possuíam endereços de memória e fluxos condicionais bem definidos para garimpar as informações sigilosas. Diante disto, é proposto uma pesquisa para responder: é possível explorar informações sensíveis de softwares cujo fluxo condicional e estrutura de memória sejam desconhecidos por meio do emprego da execução especulativa?

\subsection{Questões de Pesquisa}
Questões abordadas por esta pesquisa:
        1. Quais informações, como fluxo condicional, estrutura de memória e etc; são necessárias para se aplicar execução especulativa em um software?
        2. Além de usar o tempo de retorno da memória cache para extrair as informações da própria memória cache, quais ataques de hardware também podem ser usados para desviar dados deste canal lateral?
        3. Quais endereços da memória cache são acessíveis a um programa de nível de usuário, que podem serem utilizados como canal alternativo?
        4. Quais são os outros canais laterais existentes (além da memória cache), que outros autores já exploraram?
Questões de pesquisa para trabalhos futuros:
        1. Como executar instruções mais complexas, além de somente desvios de informação, aproveitando-se da execução especulativa?
        2. Segundo Jann Horn et al. até mesmo códigos que não contenham instruções com ramificações condicionais estão em risco. Como explorá-los?

\subsection{Justificativa}
O grupo de segurança de TI da Google, Project Zero (GPZ), em uma recente descoberta (HORN, 2018), relatou duas novas classes de vulnerabilidades na arquitetura dos processadores de design moderno das fabricantes: Intel, AMD e ARM (comunicado as fabricantes em 1º de junho de 2017 e divulgado ao público em 03 de janeiro de 2018). Essas duas falhas de segurança batizadas de Spectre e Meltdown, chamam a atenção pois, segundo o estudo detalhado feito pelo grupo, elas afetaram processadores criados desde de 1995, das últimas duas décadas e mais de 230 chips irão continuar com a falha, segundo o Microcode Revision Guidance (CORPORATION, 2018b).

Em uma nota oficial a Intel se pronunciou evidenciando seu comprometimento em mitigar as falhas, estudando e disponibilizando firmwares para corrigi-las (ou atenuá-las). Se defendendo também de especulações em relação ao desempenho dos processadores, depois da correção via software, afirmou que qualquer diferença no desempenho “depende da carga de trabalho e, para o usuário médio do computador, não deve ser significativo e será atenuado com o tempo” (CORPORATION, 2018a). Depois disso, publicou um informativo de segurança (CORPORATION, 2018c), informando a lista de processadores da marca atingidos pelas falhas e suas ramificações.

As outras fabricantes (AMD e ARM), em parceria com algumas montadoras e produtoras de softwares também se pronunciaram em notas oficiais, assumindo ou não as falhas em seus produtos e, tomando certa medida de prevenção, formaram parcerias para lançarem correções em aplicações para usuário final (browsers por exemplo), que previnem a exploração das falhas. Fabricantes de jogos que utilizam os processadores AMD ou ARM, como Sony e Nintendo, não se pronunciaram, conforme lista oficial publicada pela Universidade de Tecnologia de Graz (TECHNOLOGY, 2018).

Depois de uma grande comoção dos fabricantes e da comunidade de software livre em busca de soluções, os autores da Universidade Católica de Leuven e os autores do IT de Israel, das Universidades de Michigan e Adelaide e da CSIRO Data61, fizeram novas publicações comprovativas de duas novas derivações da Spectre. Batizadas de Foreshadow 3 e Foreshadow-NG, em processadores de servidores e em máquinas virtuais (BULCK et al., 2018a).

Tal assunto adquiriu grande importância na divulgação pelos veículos de comunicação, pois é provável que muitas máquinas, aparelhos móveis, datacenters e consequentemente informações sigilosas sejam expostas e continuem sendo. Isto se dá, porque tal categoria de vulnerabilidade está em nível de hardware, e correções via software são paliativos que custam questões de desempenho.

\subsection{Objetivos}
Em razão dos fatos supramencionados, esta pesquisa visa esclarecer, explicar e aplicar métodos desenvolvidos pelos autores citados anteriormente para expor informações de forma genérica (em softwares comuns) em computadores pessoais. Assim, o Objetivo Geral desta pesquisa é demonstrar como a execução especulativa pode ocorrer em softwares cujo o fluxo condicional e estrutura de memória se desconhece. 
Para alcançar este Objetivo Geral, os seguintes Objetivos Específicos foram levantados: 
    Demonstrar e explorar a falha de execução especulativa em fluxos condicionais conhecidos;
    Demonstrar e explorar a falha de execução especulativa em fluxos condicionais desconhecidos;
    Demonstrar e explorar a falha de execução especulativa em regiões de memória cujos endereços não são conhecidos inicialmente.

\subsection{Procedimentos Metodológicos}
Esta é uma pesquisa com finalidade de natureza básica, com objetivos de caráter descritivos, que utiliza uma abordagem quantitativa, com procedimentos fundamentados em pesquisa de desenvolvimento, juntamente com análises e testes.

Trata-se de uma pesquisa básica porque além de promover o aprofundamento de um conhecimento já exposto através de pesquisas anteriores, não se descarta a possibilidade de produção de um conhecimento útil para estudos futuros sobre o assunto. Sendo assim, o autor espera complementar alguns aspectos e peculiaridades de pesquisas anteriormente feitas, preenchendo lacunas de conhecimento a respeito da execução especulativa.

É uma pesquisa descritiva, pois tem por objetivo demonstrar a possibilidade de execução da falha de especulação em circunstâncias específicas e diferentes das circunstâncias abordadas no artigo de Jann Horn. Essa comparação leva em conta artigos de outros autores que tentaram defender o processador contra o ataque utilizando outros métodos via software. Desta forma, existe uma associação entre a proteção para determinadas situações e o ataque, com determinadas circunstâncias (endereços de memória, softwares a serem explorados e etc).

A pesquisa tem abordagem quantitativa, visto que a avaliação dos resultados possui caráter bem definido e exato do ponto de vista funcional. Avaliando a execução e não execução dos procedimentos que levam a falha e consequentemente a invasão, é possível concluir, de forma objetiva e direta, se a técnica funciona ou não. Tais resultados serão comprovados através de testes.

A pesquisa terá seus procedimentos e resultados técnicos analisados e avaliados em máquinas virtuais e reais, induzidas ao ambiente que se é esperado para a realização dos testes. Portanto, os testes serão analisados e testados mais de uma vez, em ambientes diferentes conforme o desenvolvimento da exploração feita pelo autor.

\subsection{Organização da Monografia}

% Finaliza a parte no bookmark do PDF, para que se inicie o bookmark na raiz
\bookmarksetup{startatroot}
% ------------------------------------------------------------------------
% CONCLUSÃO
% ------------------------------------------------------------------------
%\section{Considerações finais}
%\lipsum[31]
% ------------------------------------------------------------------------
% ELEMENTOS PÓS-TEXTUAIS
% ------------------------------------------------------------------------
\postextual
% ------------------------------------------------------------------------
% REFERẼNCIAS
% ------------------------------------------------------------------------
\bibliography{referencias}
% ------------------------------------------------------------------------
% APÊNDICES
% ------------------------------------------------------------------------
\begin{apendicesenv}
%\chapter{Cras non urna sed feugiat cum sociis natoque penatibus et magnis dis parturient montes nascetur ridiculus mus}
%\lipsum[31]
\end{apendicesenv}
% ------------------------------------------------------------------------
% ANEXOS
% ------------------------------------------------------------------------
\begin{anexosenv}
\vspace{\onelineskip}
%\chapter{Cras non urna sed feugiat cum sociis natoque penatibus et magnis dis parturient montes nascetur ridiculus mus}
%\lipsum[31]
\end{anexosenv}
% ------------------------------------------------------------------------
% AGRADECIMENTOS
% ------------------------------------------------------------------------
\section*{Agradecimentos}
%\lipsum[31]
% ------------------------------------------------------------------------
% FINAL DO DOCUMENTO
% ------------------------------------------------------------------------
\end{document}