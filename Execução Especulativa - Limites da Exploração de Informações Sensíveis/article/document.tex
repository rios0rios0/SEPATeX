% ------------------------------------------------------------------------
% ------------------------------------------------------------------------
% Modelo de Artigo Acadêmico
% Em conformidade com:
% ABNT NBR 6022:2018: Informação e Documentação - Artigo em Publicação Periódica Científica - Apresentação
%
% Adaptado para:
% SEPA: Seminário Estudantil de Produção Acadêmica da UNIFACS
%
% Baseado na Biblioteca abnTeX2 v1.9.7
% ------------------------------------------------------------------------
% ------------------------------------------------------------------------
\documentclass[
	% Opções da classe memoir
	article,			    % indica que é um artigo acadêmico
	12pt,				    % tamanho da fonte
	oneside,			    % para impressão apenas no recto. Oposto a twoside
	a4paper,			    % tamanho do papel. 
	% Opções da classe abntex2
	chapter=TITLE,		    % títulos de capítulos convertidos em letras maiúsculas
	section=TITLE,		    % títulos de seções convertidos em letras maiúsculas
	subsection=TITLE,	    % títulos de subseções convertidos em letras maiúsculas
	%subsubsection=TITLE    % títulos de subsubseções convertidos em letras maiúsculas
	% Opções do pacote babel
	english,			    % idioma adicional para hifenização
	brazil,				    % o último idioma é o principal do documento
	sumario=tradicional
]{abntex2}
% ------------------------------------------------------------------------
% PACOTES
% ------------------------------------------------------------------------
% Pacotes fundamentais 
\usepackage{times}			    % Usa a fonte Times New Roman
\usepackage[T1]{fontenc}		% Selecao de codigos de fonte.
\usepackage[utf8]{inputenc}		% Codificacao do documento (conversão automática dos acentos)
\usepackage{indentfirst}		% Indenta o primeiro parágrafo de cada seção.
\usepackage{nomencl} 			% Lista de simbolos
\usepackage{color}				% Controle das cores
\usepackage{graphicx}			% Inclusão de gráficos
\usepackage{microtype} 			% Para melhorias de justificação
% Pacotes adicionais, usados apenas no âmbito do Modelo Canônico do abnteX2
\usepackage{lipsum}				% Para geração de dummy text
% Pacotes de citações
\usepackage[brazilian,hyperpageref]{backref}	 % Paginas com as citações na bibliografia
\usepackage[alf,bibjustif,abnt-emphasize=bf,abnt-etal-text=emph]{abntex2cite}  % Citações padrão ABNT, forçar a justificação da bibliografia e enfatizar com negrito
% Pacotes extras 
\usepackage{fancyhdr}           % Personalização do cabeçalho e rodapé
\usepackage{listings}           % Personalização de código
\usepackage{listings/c/style}   % Incluí estilo customizado para linguagem C
% ------------------------------------------------------------------------
% CONFIGURAÇÃO DOS PACOTES
% ------------------------------------------------------------------------
% Configurações do pacote backref
% Usado sem a opção hyperpageref de backref
\renewcommand{\backrefpagesname}{Citado na(s) página(s):~}
% Texto padrão antes do número das páginas
\renewcommand{\backref}{}
% Define os textos da citação
\renewcommand*{\backrefalt}[4]{
	\ifcase #1
		Nenhuma citação no texto.
	\or
		Citado na página #2.
	\else
		Citado #1 vezes nas páginas #2.
	\fi
}
% Configuração dos nomes padrões do babel
%\addto\captionsbrazil{
%    \renewcommand{\bibname}{Referências Bibliográficas}
%}
% Configuração do título das referências (anteriormente modificado)
\renewcommand{\bibsection}{%
    \section*{\bibname}
    \bibmark
    \ifnobibintoc\else
        \phantomsection
    \fi
    \prebibhook
}
% Modificando o tamanho da fonte "large" que é 14.4pt, para 14pt
%\renewcommand{\large}{\fontsize{14}{14}\selectfont}
% Modificando título das listagens de código
\renewcommand{\lstlistingname}{Figura}
% ------------------------------------------------------------------------
% DADOS DO DOCUMENTO
% ------------------------------------------------------------------------
% Informações de dados para capa
\autor{\normalsize{\textbf{Felipe Rios da Silva Cordeiro}}}
\instituicao{
    UNIFACS - UNIVERSIDADE SALVADOR
    ESCOLA DE ARQUITETURA, ENGENHARIA\\ E TECNOLOGIA DA INFORMAÇÃO
    BACHARELADO EM ENGENHARIA DA COMPUTAÇÃO
}
\titulo{\uppercase{\normalsize{\textbf{EXECUÇÃO ESPECULATIVA:\\
LIMITES DA EXPLORAÇÃO DE INFORMAÇÕES SENSÍVEIS}}}}
\local{Salvador}
%\data{2019}
\tipotrabalho{Trabalho de Conclusão de Curso, Graduação}
\preambulo{Trabalho de conclusão de curso apresentado ao curso de graduação em Engenharia da Computação da Universidade Salvador - UNIFACS, como requisito fundamental para obtenção do título de Engenheiro da Computação.}
\orientador[Orientador:]{\normalsize{\textbf{Éldman de Oliveira Nunes}}}
%\tituloestrangeiro{Canonical article template in \abnTeX: optional foreign title}
% ------------------------------------------------------------------------
% META DADOS DO PDF
% ------------------------------------------------------------------------
% Alterando o aspecto da cor azul
\definecolor{blue}{RGB}{41,5,195}
% Informações do PDF
\makeatletter
\hypersetup{
 	%pagebackref=true,
	pdftitle={\@title}, 
	pdfauthor={\@author},
	pdfsubject={\imprimirpreambulo},
    pdfcreator={LaTeX with abnTeX2 and Overleaf},
	pdfkeywords={abnt}{latex}{abntex}{abntex2}{atigo científico}, 
	colorlinks=false,       % false: boxed links; true: colored links
	linkcolor=blue,         % color of internal links
	citecolor=blue,        	% color of links to bibliography
	filecolor=magenta,      % color of file links
	urlcolor=blue,
	bookmarksdepth=4
}
\makeatother
% ------------------------------------------------------------------------
% CONFIGURAÇÕES DAS FOLHAS E AJUSTES NAS FONTES GERAIS
% ------------------------------------------------------------------------
% Compila o índice
\makeindex
% Altera as margens
\setlrmarginsandblock{3cm}{2cm}{*}
\setulmarginsandblock{3cm}{2cm}{*}
\checkandfixthelayout
% Espaçamentos entre linhas e parágrafos 
% O tamanho do parágrafo é dado por (espaçamento na primeira linha):
\setlength{\parindent}{1.25cm}
% O espeçamento padrão é definido como \OnehalfSpacing, ou seja, um espaço e meio conforme estabelece a ABNT NBR 14724:2011
% ------------------------------------------------------------------------
% CABEÇALHOS E RODAPÉS
% ------------------------------------------------------------------------
% Criar um novo estilo de cabeçalhos e rodapés
\pagestyle{fancy}
%\setlength{\headheight}{80pt}
\fancyhf{}
%\lhead{\includegraphics[width=0.4\textwidth]{/images/image00.png}}
%\rhead{
%{\fontsize{8}{1.5}\selectfont
%\begin{vplace}
%TCC - TRABALHO DE CONCLUSÃO DE CURSO\break
%COORDENAÇÂO DE ENGENHARIA DA COMPUTAÇÃO\end{vplace}}}
\fancypagestyle{plain}{
    \renewcommand{\headrulewidth}{0pt}
    \renewcommand{\footrulewidth}{0pt}
    \fancyhfoffset[LE]{0mm}
    \fancyhfoffset[RE]{0mm}
    \fancyhfoffset[LO]{0mm}
    \fancyhfoffset[RO]{0mm}
}
% ------------------------------------------------------------------------
% CAPÍTULOS, SEÇÕES E SUBSEÇÕES
% ------------------------------------------------------------------------
% Chapter 12pt + Bold
\renewcommand{\ABNTEXchapterfont}{\bfseries}
\renewcommand{\ABNTEXchapterfontsize}{\normalsize}
% Section 12pt + Bold
\renewcommand{\ABNTEXsectionfont}{\bfseries}
\renewcommand{\ABNTEXsectionfontsize}{\normalsize}
% SubSection 12pt
\renewcommand{\ABNTEXsubsectionfont}{\normalfont}
\renewcommand{\ABNTEXsubsectionfontsize}{\normalsize}
% SubSubSection 12pt + Bold + Underline
\renewcommand{\ABNTEXsubsubsectionfont}{\bfseries}
\renewcommand{\ABNTEXsubsubsectionfontsize}{\normalsize}
\setsubsubsecheadstyle{\ABNTEXsubsubsectionfont\ABNTEXsubsubsectionfontsize\ABNTEXsubsubsectionupperifneeded\coloruline[black]}
% SubSubSubSection 12pt + Lowercase
\setparaheadstyle{\normalfont\ABNTEXsubsubsectionfont\ABNTEXsubsubsectionfontsize}
% Retirando espaçamentos antes dos capítulos
\setlength{\beforechapskip}{\baselineskip}
% Retirando espaçamentos depois dos capítulos
\setlength{\afterchapskip}{\baselineskip}
% Recriando a variável que instancia o resumo
\renewenvironment{resumoumacoluna}{}

% ------------------------------------------------------------------------
% INÍCIO DO DOCUMENTO
% ------------------------------------------------------------------------
\begin{document}
% Seleciona o idioma do documento (conforme pacotes do babel)
%\selectlanguage{english}
\selectlanguage{brazil}
% Retira espaço extra obsoleto entre as frases.
\frenchspacing 
% ------------------------------------------------------------------------
% ELEMENTOS PRÉ-TEXTUAIS
% ------------------------------------------------------------------------
\pretextual
\pagestyle{fancy}
% página de titulo principal (obrigatório)
%\maketitle
\begin{SingleSpace}
    \begin{center}
        \imprimirtitulo
    \end{center}
    \begin{flushright}
        \imprimirautor
        \footnote{Graduando em Engenharia da Computação, UNIFACS. E-mail: felipe.rios.silva@outloook.com}
        \\
        \imprimirorientador
        \footnote{Doutor em Computação, UNIFACS. E-mail: eldman.nunes@unifacs.br}
    \end{flushright}
\end{SingleSpace}
% Titulo em outro idioma (opcional)
% Resumo em Português
\begin{resumoumacoluna}
    \footnotesize{\begin{SingleSpace}
        \noindent
        \textbf\resumoname\\
        As falhas de segurança se tornam mais preocupantes quando a segurança de informações pessoais correm risco. As descobertas de Jann Horn e de outros grupos de pesquisadores colocaram a prova a segurança de vários processadores modernos. Batizada de \emph{Spectre}, a vulnerabilidade que se aproveita da técnica de execução especulativa, está presente em quase todos os dispositivos computacionais que executam funções ao mesmo tempo. Mas, até que ponto as informações pessoais dos usuários estão expostas? Quais são os limites da exploração desta vulnerabilidade? Utilizando procedimentos técnicos e testes fundamentados, esta pesquisa esclarece os limites fundamentais que precisam ser atendidos para um vazamento ocorrer através da exploração desta vulnerabilidade. Com a documentação e o código criado pelos pesquisadores supracitados, possíveis limites de exploração da vulnerabilidade foram elicitados e provados. Estas provas levaram a uma conclusão com perspectivas futuras de estudo e evidencias que apontam para outras pesquisas mais aprofundadas.\\\\
        \textbf{Palavras-chave:} execução; especulativa; \emph{Spectre}; limites; exploração; informações.
        \vspace{\onelineskip}
    \end{SingleSpace}}
\end{resumoumacoluna}
% Resumo em Inglês
\renewcommand{\resumoname}{Abstract}
\begin{resumoumacoluna}
    \footnotesize{\begin{SingleSpace}
        \begin{otherlanguage*}{english}
            \noindent
            \textbf\resumoname\\
            Security flaws become more worrisome when personal information security is at risk. The findings of Jann Horn and other groups of researchers tested the safety of several modern processors. Specter-named vulnerability, which takes advantage of the speculative execution technique, is present in almost all computing devices that perform functions at the same time. But to what extent do user's personal information are exposed? What are the limits of exploiting this vulnerability? Using technical procedures and substantiated tests, this research clarifies the fundamental limits that need to be met for a leak to occur through the exploitation of this vulnerability. With the documentation and code created by the aforementioned researchers, possible limits of exploitation of vulnerability have been elicited and proven. These evidence led to a conclusion with future study prospects and evidence pointing to further research.\\\\
            \textbf{Keywords:} speculative; execution; \emph{Spectre}; limits; exploitation; information.
        \end{otherlanguage*}
    \end{SingleSpace}}
\end{resumoumacoluna}
% ------------------------------------------------------------------------
% ELEMENTOS TEXTUAIS
% ------------------------------------------------------------------------
\textual
\pagestyle{fancy}
% ------------------------------------------------------------------------
% INTRODUÇÃO
% ------------------------------------------------------------------------
\section{Introdução}
% ---------------------------------
% TEMA
% ---------------------------------
A execução especulativa é uma técnica de projeto de microarquitetura, que proporciona o aprimoramento da velocidade de processamento nas CPU's modernas. Esta técnica consiste na estimativa e execução de instruções com valores ainda não conhecidos pela CPU, durante o curto período de carregamento dos valores reais. Do ponto de vista funcional, esta especulação traria problemas se os resultados de especulações incorretas fossem efetivados. Porém, quando a verdadeira informação é recuperada, a CPU verifica a exatidão da suposição e descarta o ``caminho'' (fluxo de execução) que foi executado incorretamente, eliminando os valores nos registradores ou alterações em variáveis, por exemplo \cite{Intel2016Architectures}.

Utilizando ataques que combinam a indução da execução especulativa e canais laterais, Jann Horn (\citeyear{Jann2018Reading}) e outros pesquisadores provaram que é possível recuperar informações privilegiadas (que foram especuladas), através da memória cache (um canal lateral), que não é revertida quando uma especulação errônea acontece \cite{Kocher2018Spectre}. A partir disto, com o conhecimento microarquitetural necessário, é possível um atacante induzir a execução especulativa errônea, por ``viciar'' o processador em uma cadeia de especulações e transferir as informações especuladas para um canal alternativo que pode ser lido posteriormente.

% ---------------------------------
% JUSTIFICATIVA
% ---------------------------------
Estes ataques que utilizam a combinação das duas técnicas acontecem em condições específicas e precisam de circunstâncias bem definidas para serem efetivos. Mesmo assim, a abrangência destes ataques é considerável: é possível afirmar que praticamente todos os chips de arquitetura \emph{Intel}, \emph{AMD} e \emph{ARM}\footnote{\emph{Advanced RISC Machine}, ou ``Máquina Avançada de \emph{Instruções Reduzidas}'' (tradução e grifo do autor).}, foram afetados. Servidores na nuvem, celulares, \emph{desktops}, \emph{notebooks} e basicamente todos os chips que podem manter muitas instruções em execução, podem ter dados sigilosos comprometidos \cite{Graz2018Meltdown}. Segundo um guia de revisão\footnote{\emph{Microcode Revision Guidance}, publicado em abril de \citeyear{Intel2018Microcode}.} da \emph{Intel}, 193 processadores irão continuar com a falha, pois apresentaram instabilidade com as mitigações da vulnerabilidade.

Esta abrangência de processadores atingidos, torna esta vulnerabilidade um assunto relevante devido a probabilidade de que muitas máquinas, aparelhos móveis, \emph{datacenters} e consequentemente informações sigilosas estejam expostas. Isto acontece porque, tal categoria de vulnerabilidade se encontra em nível de \emph{hardware} e correções via \emph{software} são paliativos que custam questões de desempenho e estabilidade.

% ---------------------------------
% PROBLEMA E QUESTÃO DE PESQUISA
% ---------------------------------
Considerando que os pesquisadores citados utilizaram \emph{softwares} com estruturas conhecidas (endereços de memória, tamanho e conteúdo das variáveis por exemplo) em seus experimentos, algumas questões ainda permanecem em aberto. Diante disto, é possível explorar informações sensíveis de \emph{softwares} cuja estrutura de memória seja desconhecida, por meio da exploração da execução especulativa induzida, utilizando canais laterais?

% ---------------------------------
% OBJETIVOS
% ---------------------------------
Através deste questionamento, esta pesquisa visa esclarecer, explicar e aplicar técnicas desenvolvidas pelos pesquisadores supramencionados. Visando elencar e comprovar as circunstâncias envolvidas e que realmente são necessárias para se executar vazamentos de informações sensíveis. Tendo como foco, a exploração da execução especulativa em \emph{softwares} cujas estruturas se desconhece. De forma a solidificar limitações desta técnica de exploração em particular.

% ---------------------------------
% PROCEDIMENTOS METODOLÓGICOS
% ---------------------------------
E para a obtenção de êxito nesta busca por limitações da exploração destes dados sensíveis, esta pesquisa teve seus procedimentos e resultados técnicos analisados e avaliados em uma máquina real, induzida ao ambiente ideal para a realização dos experimentos. Portanto, cada experimento foi testado de forma exaustiva neste mesmo ambiente induzido. O intuito disto é promover o aprofundamento de um conhecimento já exposto através de pesquisas anteriores, sem descartar a possibilidade de produção de um conhecimento novo, que pode ser aplicado em estudos futuros sobre o assunto. Sendo assim, o autores esperam complementar alguns aspectos e peculiaridades de pesquisas anteriormente feitas, preenchendo lacunas de conhecimento a respeito da exploração da execução especulativa.

% ---------------------------------
% ORGANIZAÇÃO DO ARTIGO
% ---------------------------------
Para melhor segmentação e explanação do conhecimento, as outras seções do artigo estão dispostas como se segue: a seção 2, \textbf{referencial teórico}, aborda os conceitos de execução fora de ordem, \emph{branch condition}, execução especulativa, \emph{branch prediction}, \emph{transient instructions}, canais laterais e hierarquia de memórias cache. A seção 3, \textbf{referencial metodológico}, classifica a pesquisa e detalha os procedimentos metodológicos empregados. A seção 4, \textbf{resultados e discussões}, apresenta os experimentos e análises feitas a partir de códigos de outros pesquisadores e artigos relacionados, visando elicitar e testar as circunstâncias necessárias para o funcionamento da técnica descrita. E a seção 5, apresenta a \textbf{conclusão}, discute os resultados alcançados, considerando limitações, perspectivas futuras e conclusões finais.

% ------------------------------------------------------------------------
% REFERENCIAL TEÓRICO
% ------------------------------------------------------------------------
\section{Referencial Teórico}
Com o objetivo de esclarecer e contextualizar o tema desta pesquisa, esta sessão aborda definições dos conceitos a respeito da execução especulativa e complementos essenciais para o entendimento dos ataques que utilizam canais laterais. Além de abordar a trajetória histórica das descobertas de outros pesquisadores.

% ---------------------------------
% CONTEXTUALIZAÇÃO
% ---------------------------------
\subsection{Contextualização}
Dois grupos compostos respectivamente por: Jann Horn (\emph{Google Project Zero}); Paul Kocher em colaboração com Daniel Genkin (\emph{University of Pennsylvania} e \emph{University of Maryland}), Mike Hamburg (\emph{Rambus}), Moritz Lipp (\emph{Graz University of Technology}) e Yuval Yarom (\emph{University of Adelaide} e \emph{Data61}), em descobertas independentes (precedidos por Horn), relataram uma vulnerabilidade que afeta de forma abrangente arquiteturas \emph{Intel}, \emph{AMD} e \emph{ARM}.

Esta vulnerabilidade (comunicada aos fornecedores em 01 de Junho de 2017) que foi divulgada por Horn ao público (03 de Janeiro de \citeyear{Jann2018Reading}) com o nome de \emph{Spectre}\footnote{``Espectro'' (tradução do autor). Conforme \citeonline{Graz2018Meltdown}, o nome deriva da causa da falha.}, foi registada na CVE\footnote{\emph{Common Vulnerabilities and Exposures}, ``Exposições e Vulnerabilidades Comuns'' (tradução do autor). Mantida sem fins lucrativos pela Mitre \emph{Corporation}.}, na forma de duas variações: \emph{bypass} de verificação de limites (\emph{CVE-2017-5753}) e injeção do caminho alvo (\emph{CVE-2017-5715}). Estas duas variações foram documentadas cientificamente por \citeonline{Kocher2018Spectre}. Apenas a primeira variação será estudada nesta pesquisa. Isto porque, os ataques que a utilizam apresentam uma abrangência maior de alvos, do que até mesmo as outras variações e vulnerabilidades que sucederam esta descoberta de Horn.

Nesta mesma publicação de Horn (\citeyear{Jann2018Reading}) uma terceira variação de vulnerabilidade, que é também explorada utilizando canais laterais, foi divulgada e batizada de \emph{Meltdown}. Por sua vez, foi descoberta em paralelo por outro grupo de pesquisadores que sucedeu Horn \cite{Graz2018Meltdown}. Possui apenas um registro na CVE (\emph{CVE-2017-5754}) e tem uma abrangência de alvos limitada: processadores de arquitetura \emph{Intel} desde 1995 (exceto \emph{Intel Itanium} e \emph{Intel Atom} antes de 2013) e alguns de arquitetura \emph{ARM} \cite{Graz2018Meltdown}. Foi documentada cientificamente por \citeonline{Lipp2018Meltdown}.

Enquanto estas duas comprovações de Horn não haviam sido divulgadas, Anders Fogh (analista de \emph{malware} da empresa alemã \emph{GData}), em 28 de julho de \citeyear{Anders2017Reading}, elucidou a possibilidade de exploração da técnica de execução especulativa para ler a regiões privilegiadas de memória. Porém, obteve um resultado negativo em sua pesquisa, mantendo aberto um precedente para outros pesquisadores que posteriormente confirmariam suas suspeitas \cite{Andy2018Triple}.

Após as divulgações públicas, a \emph{Intel} comunicou em nota oficial o seu comprometimento em mitigar as falhas, estudando e disponibilizando \emph{firmwares} para corrigi-las (ou atenuá-las). Se defendendo também de especulações em relação ao desempenho dos processadores, depois das correções via \emph{software}, afirmou que qualquer diferença no desempenho ``depende da carga de trabalho e, para o usuário médio do computador, não deve ser significativo e será atenuado com o tempo'' \cite{Intel2018NewsIssues}. Depois disso, publicou um informativo de segurança, informando a lista de processadores da marca atingidos pelas vulnerabilidades e suas ramificações \cite{Intel2018SA00088}.

Outros fabricantes e fornecedores de aplicações diretamente afetados pelas vulnerabilidades também se mobilizaram, publicando informativos de segurança e lançando correções a nível de \emph{software}. Fabricantes de \emph{consoles} de jogos que utilizam processadores \emph{AMD} ou de arquitetura \emph{ARM}, como \emph{Sony} e \emph{Nintendo}, não se pronunciaram, conforme lista oficial publicada pela Universidade de Tecnologia de Graz \cite{Graz2018Meltdown}.

% ---------------------------------
% CONCLUSÃO PARCIAL
% ---------------------------------
Tanto na publicação de Jann Horn (\citeyear{Jann2018Reading}), quanto no artigo de \citeonline{Kocher2018Spectre}, as duas variações da \emph{Spectre} foram explicadas de forma bem detalhada. Porém, algumas questões ainda permanecem sem esclarecimento. Dada a base teórica mencionada, é possível entender como o ataque que se aproveita da primeira variação funciona.

A exploração do \emph{bypass} de verificação de limites consiste em “treinar” o processador a predizer o resultado de uma instrução de desvio (\emph{branch condition}). De forma que, este acerte determinado número de vezes e depois cometa uma ```violação'' proposital com valores não permitidos. A especulação desta violação, causa alterações nos componentes micro arquiteturais. E como já dito, não existe impacto na execução do \emph{software}, pois os resultados destas especulações incorretas são revertidos (de forma esperada). Porém, um dos canais micro arquiteturais que foram modificados é a memória cache. Ela contém o endereço da informação que, em condições normais, não poderia ser acessada. Durante esta pesquisa, não foi possível precisar (nem por testes, nem por literatura) qual o nível de cache esta informação é armazenada.

A literatura afirma de forma precisa que existem possíveis variações desta mesma técnica \cite{Kocher2018Spectre}. Porém, nenhuma informação adicional sobre isto foi dada. De forma semelhante, afirma-se que com a primeira variação da \emph{Spectre} é possível realizar o vazamento de informações de lugares arbitrários na memória. E obter informações até mesmo de outros \emph{softwares} que estão na mesma máquina. Porém, nenhuma prova de conceito foi divulgada até o momento (finalização desta pesquisa em junho de 2019) demonstrando tal capacidade.

Pensando nestas limitações e em informações que ainda continuam sem esclarecimento, as próximas seções segmentam metodologias e procedimentos realizados pelos autores, visando elucidar tais informações.

% ---------------------------------
% TRABALHOS RELACIONADOS
% ---------------------------------
\subsection{Trabalhos Relacionados}
As descobertas supramencionadas abriram precedentes para outros questionamentos em relação a segurança dos processadores atuais. Questionamentos sobre quantas vulnerabilidades críticas como estas, podem estar ``adormecidas'' por anos, preocupam os pesquisadores \cite{Andy2018Triple}. E motivados por estas preocupações, outras vulnerabilidades foram descobertas, partindo do mesmo princípio de ataques que se aproveitam da execução especulativa. Em ordem cronológica, foram documentadas: \emph{Foreshadow} em \citeonline{VanBulck2018Foreshadow} e \emph{Foreshadow-NG} em \citeonline{Weisse2018ForeshadowNG}. \emph{PortSmash} em \citeonline{Alejandro2018Port}. \emph{Spoiler}, em \citeonline{Islam2019Spoiler}. \emph{ZombieLoad} em \citeonline{Schwarz2019ZombieLoad}. \emph{RIDL} em \citeonline{VanSchaik2019RIDL} e \emph{Fallout} em \citeonline{Minkin2019Fallout}.

A \emph{Spectre}, que dada a cronologia de descoberta pode ser chamada de ``pioneira'' da execução especulativa, é complexa de ser explorada e possui circunstâncias (ou condições) específicas para exploração. Por isso, pesquisadores continuam a desenvolver análises e outras variações de ataques utilizando a \emph{Spectre} como base. Exemplo recente disto é: a análise publicada por \citeonline{McIlroy2019SpectreIH}; e a variação de ataque utilizando o RSB (\emph{Return Stack Buffer}) como canal micro arquitetural, que foi documentada por \citeonline{Esmaeil2018Spectre}.

Até o momento de finalização desta pesquisa (junho de 2019), não havia sido descoberta uma solução totalmente eficaz contra a vulnerabilidade. Um grande esforço estava sendo feito para o desenvolvimento de medidas de prevenção via \emph{software}, para dificultar a exploração e aumentar a robustez das aplicações \cite{Graz2018Meltdown}.

% ---------------------------------
% FUNDAMENTAÇÃO TEÓRICA
% ---------------------------------
\subsection{Fundamentação Teórica}
As funções que se encontram ordenadas em um fluxo para execução de um \emph{software} qualquer, de forma geral, são convertidas pelos processadores para instruções. E, dentro do núcleo do processador, estas instruções por sua vez, são convertidas para micro-operações \cite{Alisson2017Introducao}. A execução destas instruções pelo processador, acontece fora de ordem em grande parte dos momentos. Este paradigma é chamado de \textbf{execução fora de ordem} (\emph{out-of-order execution}) \cite{Fog2017Microarchitecture}.

É um paradigma que aumenta a utilização dos componentes do processador. Permite que instruções que sucedem o fluxo atual de execução, sejam executadas em paralelo, ou até mesmo antes do fluxo de execução atual. E, assim que as micro-operações de uma instrução são finalizadas, bem como as instruções que a antecedem, estas instruções são retiradas, obedecendo a ordem de execução predefinida. Limpando assim o \emph{buffer} de reordenação e efetivando as alterações nos registradores (variáveis do processador) \cite{Kocher2018Spectre}.

Durante o fluxo de execução deste \emph{software} qualquer, utilizado como exemplo, podem existir instruções de desvio: condicionais ou incondicionais. Quando o processador se depara com uma instrução de desvio condicional (chamada de \textbf{\emph{branch condition}}), muitas vezes ele não conhece quais são as instruções futuras. Pois, esta instrução de ramificação tem uma direção que depende das instruções precedentes, cuja execução ainda não está concluída. 

Esta situação é claramente vista quando observa-se o endereço de memória que aponta para a próxima instrução (que continuaria o fluxo). Este endereço pode ser especificado de forma: imediata, endereço de memória fixo; direta, variável que contém o endereço de memória; ou indireta, endereço de memória de uma variável que contém o próximo salto \cite{Debarshi2018Addressing}. Quando o endereço de memória é especificado através de uma variável (de forma direta) e o conteúdo desta variável se trata do resultado de uma operação, tem-se uma condição propícia para uma ``especulação''.

Nestas situações, o processador pode preservar o estado dos registradores e variáveis, fazer uma previsão do caminho que o programa seguirá e executar instruções especulativamente ao longo deste caminho. Se a previsão estiver correta, os resultados desta execução serão confirmados (isto é, salvos nos registradores e variáveis), gerando uma vantagem de desempenho em relação ao tempo ocioso durante a espera. Caso contrário, o processador abandona o trabalho executado, revertendo o estado dos registradores e variáveis, continuando ao longo do caminho correto. Esta técnica é chamada de \textbf{execução especulativa} (\emph{speculative execution}) \cite{Kocher2018Spectre}.

Durante a execução especulativa, o processador faz suposições sobre o resultado provável das instruções de desvio. Previsões melhores consequentemente melhoram o desempenho, aumentando o número de operações executadas especulativamente que podem ser confirmadas com sucesso \cite{Kocher2018Spectre}. Esta técnica de previsão de resultados, aplicada a uma \emph{branch condition} é chamada de \textbf{\emph{branch prediction}}.

\citeonline{Kocher2018Spectre} chamou as instruções que são realizadas erroneamente em uma tentativa de execução especulativa (como resultado de uma previsão incorreta) de ``instruções transitórias'' (\textbf{\emph{transient instructions}}). Admitindo que tais instruções, embora não alterem o fluxo de execução de um \emph{software}, deixam rastros em componentes micro arquiteturais. Quando um invasor passa a coletar esses rastros através destes componentes, pode-se dizer que estes dados estão vazando através de um \textbf{canal}.

O comportamento de prever uma ação futura, baseado em ações passadas, assumindo que a ação futura é semelhante a anterior, permite que alterações no estado de micro arquitetura causadas pelo comportamento de um \emph{software} possam afetar outros \emph{softwares}. E, ao longo do tempo, a exploração destas pequenas alterações foram comprovadas de diferentes formas, elicitadas por \citeonline{Kocher2018Spectre}. As explorações de componentes micro arquiteturais (canais), são chamadas de \textbf{ataques de canais laterais} (\emph{side-channals attacks}). E, um destes canais que é conhecido e costumeiramente explorado por ataques de canais laterais, é a memória cache.

A memória cache é abordada como algo singular, mas se refere a um conjunto de memórias, hierarquicamente subdivididas e organizadas. Um processador comum, inclui uma \textbf{hierarquia de memórias caches} em sua arquitetura. A cache que fica no topo, é chamada de L1 (\emph{level} 1 ou nível 1), é a menor e mais rápida. As L2 e L3, respeitam a ordem de serem maiores entre si conforme se avança de nível e perdem também em velocidade do mesmo modo.

Segundo \citeonline{Yarom2015MappingTI} nos processadores Intel modernos, o tamanho da cache L1 é de 64 KiB, com uma velocidade de acesso de 4 ciclos de CPU. Na L2, são 256 KiB com 7 ciclos. E, em cada núcleo do processador, tem-se níveis L1 e L2 dedicados. O terceiro nível (L3) é também chamado de LLC (\emph{last-level} cache). O tamanho da LLC varia entre os processadores (de 3 MiB a mais de 20 MiB), como também sua velocidade (26 a 31 ciclos de CPU). E é compartilhada entre os núcleos de um processador que é multinúcleos (\emph{multicores}).

% ------------------------------------------------------------------------
% REFERENCIAL METODOLÓGICO
% ------------------------------------------------------------------------
\section{Referencial Metodológico}
Esta seção aborda a classificação da pesquisa com relação a finalidade, abordagem e procedimentos técnicos. Além de caracterizar e segmentar os procedimentos técnicos em passos.
% ---------------------------------
% CLASSIFICAÇÃO DA PESQUISA
% ---------------------------------
\subsection{Metodologia}
Esta é uma pesquisa com finalidade de natureza básica, com objetivos de caráter explicativos, que utiliza uma abordagem quantitativa, com procedimentos fundamentados em pesquisa bibliográfica, documental e experimental. Caracterizada por um estudo transversal, levando em conta os resultados de análises e testes conduzidos em um ambiente previamente configurado, ou seja, em um laboratório.

Trata-se de uma \textbf{pesquisa básica} \cite{Priscilla2017Metodologia} porque, além de promover o aprofundamento de um conhecimento já exposto através de pesquisas anteriores, não se descarta a possibilidade de produção de novos conhecimentos sobre o assunto. Sendo assim, os autores esperam complementar alguns aspectos e peculiaridades de pesquisas anteriormente feitas, preenchendo lacunas de conhecimento a respeito exploração da execução especulativa.

É uma \textbf{pesquisa explicativa} \cite{Rocha2016Dom}, pois tem por objetivo demonstrar a possibilidade de execução da técnica de vazamento, em circunstâncias específicas e diferentes das circunstâncias abordadas no artigo de \citeonline{Kocher2018Spectre}. Essa comparação leva em conta artigos de outros pesquisadores, que elucidaram limites que estão envolvidos na aplicação da técnica. Existindo assim, uma associação entre os limites elicitados, a comprovação ou não destes e a busca de outros limites, para a consolidação das circunstâncias necessárias.

A pesquisa utiliza uma \textbf{abordagem quantitativa} \cite{Rocha2016Dom}, visto que a avaliação dos resultados possui caráter bem definido e exato do ponto de vista funcional. Avalia-se a execução ou não dos procedimentos que levam ao vazamento e concluir-se de forma objetiva e direta se a técnica funciona ou não e com quais circunstâncias. Tais resultados foram comprovados através de experimentos, embasados na literatura referencial.

Os procedimentos técnicos adotados pelos autores baseiam-se em técnicas de pequisa bibliográfica, documental e experimental. O \textbf{procedimento bibliográfico} foi adotado, pois proporciona a condição de comparação histórica com a literatura, além de conferir o embasamento teórico veraz necessário (desconsiderando literaturas sensacionalistas, imprecisas ou sem fontes sólidas), creditando aos autores primários suas devidas contribuições para o ponto atingido deste trabalho. O \textbf{procedimento documental} confere a garantia de consulta e comparação dos artigos primários, com a literatura não processada (manuais, guias e notas oficiais dos fabricantes). E o \textbf{procedimento experimental} proporcionou aos autores agregarem valor ao trabalho em questão através de experimentos com retestes. Utilizando como objetos de estudo os \emph{softwares} desenvolvidos por outros pesquisadores, afim de obter resultados diferentes com modificações pontuais, visando o acréscimo de um conhecimento anterior \cite{Praca2015Metodologia}.

Estes procedimentos técnicos, juntamente com os resultados, foram analisados e avaliados durante um determinado período de tempo. Sendo que, as variáveis observadas são relativas a momentos instantâneos de regiões dinâmicas de memória, conferindo uma característica \textbf{transversal} ao estudo, tratando-se de tempo de aplicação \cite{Setia2016Methodology}.

A pesquisa foi conduzida em um \textbf{ambiente controlado}, com emprego de um computador pessoal. Este ambiente laboratorial foi previamente configurado, para simular a ausência de defesa contra a técnica de vazamento em questão (desprovido de atualizações recomendadas). Foi necessário testar a mesma técnica no mesmo cenário criado, de forma exaustiva, evidenciando cada limite encontrado para a consolidação das circunstâncias necessárias.

% ---------------------------------
% DESCRIÇÃO DOS PROCEDIMENTOS
% ---------------------------------
\subsection{Procedimentos Técnicos}
Para a realização dos experimentos, utilizou-se a prova de conceito (\emph{Proof of Concept}, ou \emph{PoC}) documentada por \citeonline{Kocher2018Spectre}. O código exemplo de exploração (ou \emph{exploit}\footnote{É um conjunto de instruções \cite{Cruz2016Estudo} capaz de tirar proveito de uma vulnerabilidade \cite{Cambridge2019Exploit}.}) desta \emph{PoC} foi escrito em linguagem C e simula o vazamento de uma informação secreta escrita diretamente no código. O intuito é explorar a execução especulativa induzida, utilizando a memória cache como canal lateral em uma estrutura (endereços de memória e variáveis) definida pelo próprio criador do \emph{exploit}. 

De posse deste \emph{exploit} e utilizando uma máquina convencional que contenha um compilador de C (\textbf{Visual C++} para \emph{Windows}, ou \textbf{GNU Compiler Collection} para \emph{Linux}), é possível verificar os passos que desencadeiam o vazamento. No caso desta pesquisa, o ambiente utilizado foi um Linux GNU Ubuntu 16.04.05 LTS 64-bit, Intel\textsuperscript{\tiny\textregistered} Core\textsuperscript{\tiny\texttrademark} i3 5005U @ 2.0 Ghz x 4. Diante disto, os experimentos para a realização desta pesquisa foram divididos em quatro testes (T01 à T04), que apresentam-se como possíveis circunstâncias limitadoras para a exploração da \emph{Spectre}. Tendo como foco, a utilização do \emph{exploit} em uma estrutura contrária ao documentado por \citeonline{Kocher2018Spectre}: endereços de memória desconhecidos, tamanho e conteúdo das variáveis escolhidos de forma adversa.

Dois métodos foram utilizados para elicitar estes \textbf{possíveis limites} (ou circunstâncias limitadoras): exploração da literatura, em busca de circunstancias que sejam necessárias para a execução da \emph{Spectre}; e, exploração do \emph{exploit}, em busca de limitações intuitivas que determinam o funcionamento da \emph{PoC}. Posteriormente estes foram testados, afim de identificar com as suas possíveis falhas, em quais circunstâncias gerais e de comum aceitação a \emph{Spectre} funciona ou limita-se: quanto a endereços de memória (físicos ou virtuais) que abrangem o vazamento, quanto a quantidade de dados que pode ser vazada e quanto a aplicação genérica da vulnerabilidade para leitura arbitrária de memória, sem autorização.

% ------------------------------------------------------------------------
% TESTES E RESULTADOS
% ------------------------------------------------------------------------
\section{Testes e Resultados}
Esta seção tem por objetivo expor os testes realizados de maneira sistemática, com seus resultados individuais, abrindo questionamentos a respeito do conhecimento estudado. Ao final desta seção, uma discussão apresenta uma síntese do produto gerado pelos experimentos.
% ---------------------------------
% T01
% ---------------------------------
\subsection{T01 - Fora de contexto com endereço físico}
Conforme \citeonline{Kocher2018Spectre}, os ataques utilizando \emph{Spectre} são efetivos para as informações que a vítima tem acesso. Este tipo de ataque não escala privilégios e acessa informações privilegiadas (dependendo das informações que a vítima tem acesso). Mesmo assim, a literatura referencial afirma que é possível realizar vazamentos de dados que estão fora do contexto do atacante. Limitando-se apenas aos dados que o contexto da vítima tem acesso. O objetivo deste experimento é comprovar que o \emph{exploit} da \emph{Spectre} pode realizar estes vazamentos.

Pensando nisto, os autores desenvolveram uma aplicação que serviu de ``vítima''. Expondo de forma voluntária o seu próprio PID (\emph{Process ID}, identificador único do processo para o Sistema Operacional) e o endereço virtual da variável que contém as informações secretas, conforme segue na imagem \ref{lst:01}. Esta exposição voluntária foi necessária para facilitar a busca das variáveis entre os diferentes contextos. Com o intuito de não haver maiores empecilhos na recuperação dos dados.

\lstinputlisting[language=C, style=c, caption={Aplicação vítima utilizada durante os testes.}, label={lst:01}]{listings/list01.c}

Com esta aplicação vítima e utilizando uma função (figura \ref{lst:02}\footnote{O código original pode ser encontrado em: \url{https://github.com/cirosantilli/linux-kernel-module-cheat/blob/873737bd1fc6e5ee0378f21e9df1c52e3f61e3fb/userland/virt_to_phys_user.c}}) que recebe um \emph{PID} e o endereço virtual de uma variável (neste caso a secreta), foi possível obter o endereço físico (endereço real) desta variável. Com este endereço físico em mãos e aplicado ao \emph{exploit} no lugar do cálculo de endereçamento inicial, as inferências foram iniciadas com um intervalo de 1 MB somado ao endereço físico inicial (margem de erro, para localizar o conteúdo da variável).

\lstinputlisting[language=C, style=c, caption={Função de conversão de endereço virtual para físico.}, label={lst:02}]{listings/list02.c}

Após o termino das inferências (5 minutos), foi possível perceber que não é possível inferir dados de outro contexto, utilizando o endereço físico. Porque primeiro, o sistema operacional limita o acesso, mesmo que seja de somente leitura a outras variáveis de outros contextos. E segundo, é preciso utilizar privilégios superiores aos de usuário normal para adquirir o endereço físico de uma variável qualquer de outro processo.

% ---------------------------------
% T02
% ---------------------------------
\subsection{T02 - Fora de contexto com \emph{flush} da cache}
Usando ainda a afirmação de \citeonline{Kocher2018Spectre}, a respeito da execução da \emph{Spectre} em contextos distintos, foi possível elencar outra estratégia de execução do \emph{exploit} em outros contextos. O objetivo deste experimento é testar outra possibilidade de vazamento utilizando contextos diferentes, porém, aplicando a técnica de \emph{Flush+Reload} do cache.

O \emph{exploit} foi adaptado em diversos pontos, de forma a executar somente a função de inferência dos dados. A vítima foi preparada para receber por argumento de inicialização, variáveis que podem estourar sua estrutura condicional vulnerável (conforme a figura \ref{lst:02}). O contexto atacante foi acrescido de funções para mapear os endereços físicos de duas variáveis do espaço da vítima, afim de capturar a diferença entre o vetor de exploração e o vetor secreto de caracteres. Com esta diferença em mãos, era esperado estourar o vetor de exploração e atingir caracteres secretos sem provocar sinais de \emph{SIGSEGV} na execução.

No locais onde a instrução \lstinline[language=C, style=c]{_mm_clflush} era utilizada, foi inserida uma função para realizar \emph{Flush+Reload} da memória cache, com instruções em \emph{Assembly}. Espera-se efetuar o \emph{flush} de endereços físicos de outro contexto, para a concretização da técnica, pois as chamadas exaustivas que provocarão o \emph{reload} estão funcionando de maneira teórica.

Após os testes, a aplicação atacante parou sua execução com um sinal de \emph{SIGSERV (Segmentation fault)}. Acusando a não permissão de realização do \emph{flush} de um endereço físico, fora do próprio contexto. Isto acontece pois, segundo \emph{Intrinsics Guide} da \emph{Intel}, instruções de \emph{flush} são utilizadas para limpar uma informação de mesmo contexto, em todos os níveis da hierarquia da cache \cite{Intel2018Intrinsics}. E, ao especular os dados o \emph{exploit} precisa fazer o ciclo de tentativa de acesso e limpeza da cache sucessivas vezes, inferindo o menor tempo de acesso das variáveis. Quando isto não ocorre no mesmo contexto, o \emph{exploit} não é capaz de manipular o comportamento da memória cache e portanto, não consegue realizar as inferências.

\begin{comment}
explicar mais sobre o que é "contexto" nesse sentindo
\end{comment}

% ---------------------------------
% T03
% ---------------------------------
\subsection{T03 - Limite de tamanho da cache}
As especificações da \emph{Intel} (\citeyear{Intel2019Corei3}) e \emph{CPU Wolrd} (\citeyear{CPU2016Corei3}), documentam o tamanho da hierarquia da memória cache, utilizada neste experimento. É possível confirmar estes valores utilizando o comando\footnote{Documentado nas páginas do manual do \emph{Ubuntu} (\citeyear{Ubuntu2019Lscpu}).}: \lstinline[language=C, style=c]{lscpu | grep "cache"}. Que tem como resultado os valores de cada unidade na hierarquia: \textbf{L1d: 32K} (dados), \textbf{L1i: 32K} (instruções), \textbf{L2: 256K}, \textbf{L3: 3072K}. Objetiva-se comprovar que é possível utilizar o \emph{exploit} da \emph{Spectre}, fora dos limites que a memória cache do processador em questão é capaz de suportar. Aplicando tamanhos e conteúdos dinâmicos nas variáveis. Diferente do modelo original no código do \emph{exploit}, que contém variáveis com o conteúdo estático e tamanho fixo. Será levado em conta o tamanho da cache \emph{LLC}, pois o \emph{branch predictor} pode fazer um cache \emph{hit} em qualquer nível de cache.

Antes de iniciar as modificações pontuais, o \emph{exploit} foi adaptado para ter seu tempo de execução calculado, contando a partir do inicio da fase de treinamento (conforme a figura \ref{lst:03}). Após isto, a primeira modificação feita para este experimento, foi o aumento do tamanho da informação secreta no código do \emph{exploit}. Aumentado de 40 bytes, para um valor randômico entre 3072 e 4072 Kbytes (conforme a figura \ref{lst:03}). Este vetor de caracteres foi preenchido randomicamente e submetido ao código do \emph{exploit} para inferência das variáveis. Era esperado que a memória cache removesse os valores iniciais e somente os valores finais pudessem ser inferidos.

\lstinputlisting[language=C, style=c, caption={Contagem do tempo de execução e geração do vetor randômico.}, label={lst:03}]{listings/list03.c}

Após a finalização deste experimento, concluiu-se que, ao saber o ponteiro de inicio da variável que contém a informação secreta e de posse do tamanho desta informação, é possível inferir todos os dados em uma velocidade que está num intervalo de \textbf{4 KB/s} a \textbf{7 KB/s} (para o ambiente em questão). Mesmo que a informação inferida seja maior que a memória cache. Isto porque, a transferência dos dados acontece com um endereço por vez. Cada endereço inferido passa por 3 fases: treinamento (\emph{loop} que acontece 30 vezes), leitura (\emph{loop} de 256 vezes) e inferência da melhor leitura (\emph{loop} de 256 vezes). Sendo que, cada etapa ocorre 999 vezes, garantindo uma \textbf{acurácia de 0,01\%} nos resultados, conforme \citeonline{Kocher2018Spectre}.

\begin{comment}
Adicionar uma imagem com as saídas.
\end{comment}

% ---------------------------------
% T04
% ---------------------------------
\subsection{T04 - Utilização de \emph{bit twiddling}}
A figura \ref{lst:04} contém um trecho do \emph{exploit} chamado de \emph{bit twiddling} pelos autores. Este trecho de código está presente no \emph{exploit}, para unicamente simular o funcionamento de uma estrutura condicional. Segundo \citeonline{Kocher2018Spectre}, as estruturas condicionais ativam o \emph{branch predictor} e, durante a fase de treinamento falso (\emph{mistraining}, ou treinamento errado) da memória cache, o \emph{predictor} não deveria ser chamado, pois adicionaria mais linhas na cache e, de certa forma, agregaria imprecisão a inferência do \emph{exploit}.

\lstinputlisting[language=C, style=c, caption={\emph{Bit twiddling} no código do \emph{exploit}.}, label={lst:04}]{listings/list04.c}

Trocando esta estrutura de \emph{bit twiddling}, para uma estrutura condicional convencional, espera-se provar que o grau de imprecisão agregado ao \emph{exploit} não é relevante para a checagem das informações. Sendo esta troca bem sucedida, tem-se uma variação do \emph{exploit} mais legível e fácil de se interpretar.

Concluiu-se que, após a troca das estruturas condicionais, ao ativar o \emph{branch predictor} na fase de treinamento, o \emph{BTB} perdeu as referências, o que afetou a inferência de dados do \emph{exploit}. Também foi percebido que isso só acontece no contexto atual. Pois quando o \emph{exploit} foi testado, outras atividades estavam ocorrendo na máquina alvo e isto não afetou as inferências de informações.

% ---------------------------------
% DISCUSSÃO
% ---------------------------------
\subsection{Discussão}
Com a finalização dos experimentos, foi possível comprovar que algumas condições específicas são imprescindíveis para fazer uma informação vazar através da exploração da execução especulativa. A variação 01 da \emph{Spectre} possuí circunstâncias limitadoras bem definidas e específicas, que não estavam fundamentalmente claros na literatura.

A \emph{PoC} da \emph{Spectre} acontece em contextos iguais de forma absoluta e em contextos diferentes de forma relativa. Quando \citeonline{Kocher2018Spectre} cita que "um atacante pode ler memória arbitrária de um outro contexto", ele quer dizer que esse outro contexto é subcontexto de um pai. E tanto o atacante, quanto a vítima, pertencem ao mesmo subcontexto pai. Esse é um dos limites de exploração da \emph{Spectre}.

Além de estarem no mesmo contexto pai, vítima e atacante devem compartilhar de determinada região de memória. Isto é um limite também verdadeiro. Conforme os experimentos feitos, não existe possibilidade de manipulação da memória cache, nem de cálculo de endereços de variáveis, cujo atacante não conhece ou não tem acesso. Seria necessário que a própria vítima mostrasse os endereços físicos e, mesmo assim, o sistema operacional negaria as solicitações de \emph{flush} destes endereços físicos.

% Finaliza a parte no bookmark do PDF, para que se inicie o bookmark na raiz
\bookmarksetup{startatroot}
% ------------------------------------------------------------------------
% CONCLUSÃO
% ------------------------------------------------------------------------
\section{Conclusão}
As falhas de segurança nos computadores muitas vezes preocupam as pessoas. Principalmente quando se trata da segurança de informações pessoais. As vulnerabilidades de arquitetura descobertas por Jann Horn, colocaram muitas preocupações na comunidade de pesquisa, a respeito de quão seguros são os processadores que o mundo inteiro está usando.

Por mais preocupantes que estas falhas as vezes pareçam ser, ou até mesmo, por mais vulneráveis que os computadores possam parecer, nem sempre o que se pode ganhar de vantagem com essas falhas é algo relevante. No caso da primeira variação da \emph{Spectre} (que é uma falha de nível micro arquitetural, aparentemente sem correção fácil), as condições necessárias para exploração são tão específicas que, dificilmente, alguém conseguiria explorar remotamente esta vulnerabilidade para realizar um vazamento de informações sigilosas.

Estas condições foram elucidadas através de testes, que provaram que a primeira variação da \emph{Spectre} é uma vulnerabilidade que acontece em um trecho de código padronizado e específico (uma estrutura de verificação condicional de limites). E para ser explorada, esse mesmo código vulnerável precisa estar no mesmo contexto do atacante. Que por sua vez, precisa ter permissões de acesso aos endereços de todas variáveis da vítima, que estejam envolvidas neste código vulnerável.

Foi possível através desta pesquisa delimitar as circunstâncias necessárias para a exploração da primeira variação da \emph{Spectre}. Além de realizar comparações históricas fiéis as descobertas dos pesquisadores envolvidos. Esta pesquisa também possibilitou a sistematização e segmentação de um conhecimento relativamente novo e com um alto potencial de relevância acadêmica. O que pode auxiliar pesquisadores em revisões da literatura ou até mesmo em técnicas de pesquisa para produção de novos artigos sobre o assunto.

O resultado negativo obtido nos testes de extração de dados de outros contextos pode ser considerado
como uma limitação desta pesquisa ou uma barreira técnica. Tal limitação poderá ser superada por
outras pesquisas que apresentem resultados positivos no futuro.

Como perspectiva de pesquisas futuras, pretende-se realizar novos testes para comprovar as limitações da primeira variação da \emph{Spectre}. Porém, com foco em outras formas de exploração de informações sensíveis. Não somente limitando-se a exploração de código nativo, como o feito nesta pesquisa. Mas, explorando outras possibilidades já elucidadas por Jann Horn, como por exemplo os interpretadores de código em tempo real (JIT, \emph{just-in-time}).

% ------------------------------------------------------------------------
% ELEMENTOS PÓS-TEXTUAIS
% ------------------------------------------------------------------------
\postextual
% ------------------------------------------------------------------------
% REFERẼNCIAS
% ------------------------------------------------------------------------
\bibliography{references}
% ------------------------------------------------------------------------
% APÊNDICES
% ------------------------------------------------------------------------
\begin{apendicesenv}
%\chapter{Cras non urna sed feugiat cum sociis natoque penatibus et magnis dis parturient montes nascetur ridiculus mus}
%\lipsum[31]
\end{apendicesenv}
% ------------------------------------------------------------------------
% ANEXOS
% ------------------------------------------------------------------------
\begin{anexosenv}
\vspace{\onelineskip}
%\chapter{Cras non urna sed feugiat cum sociis natoque penatibus et magnis dis parturient montes nascetur ridiculus mus}
%\lipsum[31]
\end{anexosenv}
% ------------------------------------------------------------------------
% AGRADECIMENTOS
% ------------------------------------------------------------------------
\section*{Agradecimentos}
Agradecimentos a \textbf{Anderson} Eduardo Nascimento (Co-Fundador na \emph{Allele Security Intelligence}\footnote{Acesso em: \url{https://allelesecurity.com.br}}) pelas correções de conceitos, delimitação do escopo de pesquisa e informações pertinentes ao entendimento dos conceitos fundamentais. Também a \textbf{Alan} Messias Cordeiro (alacerda), pela ajuda na escolha do tema e matérias relacionadas.
% ------------------------------------------------------------------------
% FINAL DO DOCUMENTO
% ------------------------------------------------------------------------
\end{document}