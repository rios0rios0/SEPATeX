% ------------------------------------------------------------------------
% ------------------------------------------------------------------------
% Modelo de Artigo Acadêmico
% Em conformidade com:
% ABNT NBR 6022:2018: Informação e Documentação - Artigo em Publicação Periódica Científica - Apresentação
%
% Adaptado para:
% SEPA: Seminário Estudantil de Produção Acadêmica da UNIFACS
%
% Baseado na Biblioteca abnTeX2 v1.9.7
% ------------------------------------------------------------------------
% ------------------------------------------------------------------------
\documentclass[
	% Opções da classe memoir
	article,			    % indica que é um artigo acadêmico
	12pt,				    % tamanho da fonte
	oneside,			    % para impressão apenas no recto. Oposto a twoside
	a4paper,			    % tamanho do papel. 
	% Opções da classe abntex2
	chapter=TITLE,		    % títulos de capítulos convertidos em letras maiúsculas
	section=TITLE,		    % títulos de seções convertidos em letras maiúsculas
	subsection=TITLE,	    % títulos de subseções convertidos em letras maiúsculas
	%subsubsection=TITLE    % títulos de subsubseções convertidos em letras maiúsculas
	% Opções do pacote babel
	english,			    % idioma adicional para hifenização
	brazil,				    % o último idioma é o principal do documento
	sumario=tradicional
]{abntex2}
% ------------------------------------------------------------------------
% PACOTES
% ------------------------------------------------------------------------
% Pacotes fundamentais 
\usepackage{times}			    % Usa a fonte Times New Roman
\usepackage[T1]{fontenc}		% Selecao de codigos de fonte.
\usepackage[utf8]{inputenc}		% Codificacao do documento (conversão automática dos acentos)
\usepackage{indentfirst}		% Indenta o primeiro parágrafo de cada seção.
\usepackage{nomencl} 			% Lista de simbolos
\usepackage{color}				% Controle das cores
\usepackage{graphicx}			% Inclusão de gráficos
\usepackage{microtype} 			% Para melhorias de justificação
% Pacotes adicionais, usados apenas no âmbito do Modelo Canônico do abnteX2
\usepackage{lipsum}				% Para geração de dummy text
% Pacotes de citações
\usepackage[brazilian,hyperpageref]{backref}	 % Paginas com as citações na bibliografia
\usepackage[alf,bibjustif,abnt-emphasize=bf,abnt-etal-text=emph]{abntex2cite}  % Citações padrão ABNT, forçar a justificação da bibliografia e enfatizar com negrito
% Pacotes extras 
\usepackage{fancyhdr}           % Personalização do cabeçalho e rodapé
% ------------------------------------------------------------------------
% CONFIGURAÇÃO DOS PACOTES
% ------------------------------------------------------------------------
% Configurações do pacote backref
% Usado sem a opção hyperpageref de backref
\renewcommand{\backrefpagesname}{Citado na(s) página(s):~}
% Texto padrão antes do número das páginas
\renewcommand{\backref}{}
% Define os textos da citação
\renewcommand*{\backrefalt}[4]{
	\ifcase #1
		Nenhuma citação no texto.
	\or
		Citado na página #2.
	\else
		Citado #1 vezes nas páginas #2.
	\fi
}
% Configuração dos nomes padrões do babel
%\addto\captionsbrazil{
%    \renewcommand{\bibname}{Referências Bibliográficas}
%}
% Configuração do título das referências (anteriormente modificado)
\renewcommand{\bibsection}{%
    \section*{\bibname}
    \bibmark
    \ifnobibintoc\else
        \phantomsection
    \fi
    \prebibhook
}
% Modificando o tamanho da fonte "large" que é 14.4pt, para 14pt
%\renewcommand{\large}{\fontsize{14}{14}\selectfont}
% ------------------------------------------------------------------------
% DADOS DO DOCUMENTO
% ------------------------------------------------------------------------
% Informações de dados para capa
\autor{\normalsize{\textbf{Felipe Rios da Silva Cordeiro}}}
\instituicao{
    UNIFACS - UNIVERSIDADE SALVADOR
    ESCOLA DE ARQUITETURA, ENGENHARIA\\ E TECNOLOGIA DA INFORMAÇÃO
    BACHARELADO EM ENGENHARIA DA COMPUTAÇÃO
}
\titulo{\uppercase{\normalsize{\textbf{INTELIGÊNCIA ARTIFICIAL DISTRIBUÍDA\\
E AUTOMAÇÃO INDUSTRIAL}}}}
\local{Salvador}
%\data{2019}
\tipotrabalho{Trabalho de Conclusão de Disciplina, Graduação}
\preambulo{Trabalho de conclusão de disciplina apresentado ao curso de graduação em Engenharia da Computação da Universidade Salvador - UNIFACS, como requisito fundamental para a conclusão da disciplina Automação e Instrumentação, do título de Engenheiro da Computação.}
\orientador[Orientador:]{\normalsize{\textbf{???}}}
%\tituloestrangeiro{Canonical article template in \abnTeX: optional foreign title}
% ------------------------------------------------------------------------
% META DADOS DO PDF
% ------------------------------------------------------------------------
% Alterando o aspecto da cor azul
\definecolor{blue}{RGB}{41,5,195}
% Informações do PDF
\makeatletter
\hypersetup{
 	%pagebackref=true,
	pdftitle={\@title}, 
	pdfauthor={\@author},
	pdfsubject={\imprimirpreambulo},
    pdfcreator={LaTeX with abnTeX2 and Overleaf},
	pdfkeywords={abnt}{latex}{abntex}{abntex2}{atigo científico}, 
	colorlinks=false,       % false: boxed links; true: colored links
	linkcolor=blue,         % color of internal links
	citecolor=blue,        	% color of links to bibliography
	filecolor=magenta,      % color of file links
	urlcolor=blue,
	bookmarksdepth=4
}
\makeatother
% ------------------------------------------------------------------------
% CONFIGURAÇÕES DAS FOLHAS E AJUSTES NAS FONTES GERAIS
% ------------------------------------------------------------------------
% Compila o índice
\makeindex
% Altera as margens
\setlrmarginsandblock{3cm}{2cm}{*}
\setulmarginsandblock{3cm}{2cm}{*}
\checkandfixthelayout
% Espaçamentos entre linhas e parágrafos 
% O tamanho do parágrafo é dado por (espaçamento na primeira linha):
\setlength{\parindent}{1.25cm}
% O espeçamento padrão é definido como \OnehalfSpacing, ou seja, um espaço e meio conforme estabelece a ABNT NBR 14724:2011
% ------------------------------------------------------------------------
% CABEÇALHOS E RODAPÉS
% ------------------------------------------------------------------------
% Criar um novo estilo de cabeçalhos e rodapés
\pagestyle{fancy}
%\setlength{\headheight}{80pt}
\fancyhf{}
%\lhead{\includegraphics[width=0.4\textwidth]{/images/image00.png}}
%\rhead{
%{\fontsize{8}{1.5}\selectfont
%\begin{vplace}
%TCC - TRABALHO DE CONCLUSÃO DE CURSO\break
%COORDENAÇÂO DE ENGENHARIA DA COMPUTAÇÃO\end{vplace}}}
\fancypagestyle{plain}{
    \renewcommand{\headrulewidth}{0pt}
    \renewcommand{\footrulewidth}{0pt}
    \fancyhfoffset[LE]{0mm}
    \fancyhfoffset[RE]{0mm}
    \fancyhfoffset[LO]{0mm}
    \fancyhfoffset[RO]{0mm}
}
% ------------------------------------------------------------------------
% CAPÍTULOS, SEÇÕES E SUBSEÇÕES
% ------------------------------------------------------------------------
% Chapter 12pt + Bold
\renewcommand{\ABNTEXchapterfont}{\bfseries}
\renewcommand{\ABNTEXchapterfontsize}{\normalsize}
% Section 12pt + Bold
\renewcommand{\ABNTEXsectionfont}{\bfseries}
\renewcommand{\ABNTEXsectionfontsize}{\normalsize}
% SubSection 12pt
\renewcommand{\ABNTEXsubsectionfont}{\normalfont}
\renewcommand{\ABNTEXsubsectionfontsize}{\normalsize}
% SubSubSection 12pt + Bold + Underline
\renewcommand{\ABNTEXsubsubsectionfont}{\bfseries}
\renewcommand{\ABNTEXsubsubsectionfontsize}{\normalsize}
\setsubsubsecheadstyle{\ABNTEXsubsubsectionfont\ABNTEXsubsubsectionfontsize\ABNTEXsubsubsectionupperifneeded\coloruline[black]}
% SubSubSubSection 12pt + Lowercase
\setparaheadstyle{\normalfont\ABNTEXsubsubsectionfont\ABNTEXsubsubsectionfontsize}
% Retirando espaçamentos antes dos capítulos
\setlength{\beforechapskip}{\baselineskip}
% Retirando espaçamentos depois dos capítulos
\setlength{\afterchapskip}{\baselineskip}
% Recriando a variável que instancia o resumo
\renewenvironment{resumoumacoluna}{}

% ------------------------------------------------------------------------
% INÍCIO DO DOCUMENTO
% ------------------------------------------------------------------------
\begin{document}
% Seleciona o idioma do documento (conforme pacotes do babel)
%\selectlanguage{english}
\selectlanguage{brazil}
% Retira espaço extra obsoleto entre as frases.
\frenchspacing 
% ------------------------------------------------------------------------
% ELEMENTOS PRÉ-TEXTUAIS
% ------------------------------------------------------------------------
\pretextual
\pagestyle{fancy}
% página de titulo principal (obrigatório)
%\maketitle
\begin{SingleSpace}
    \begin{center}
        \imprimirtitulo
    \end{center}
    \begin{flushright}
        \imprimirautor
        \footnote{Graduando em Engenharia da Computação, UNIFACS. E-mail: felipe.rios.silva@outloook.com}
        \\
        \normalsize{\textbf{Denisson Santos Fiaes}}\footnote{Graduando em Engenharia da Computação, UNIFACS. E-mail: dfiaes@gmail.com}
        \\
        \normalsize{\textbf{Larisse Pereira de Cerqueira}}\footnote{Graduanda em Engenharia da Computação, UNIFACS. E-mail: lare403@gmail.com}
    \end{flushright}
\end{SingleSpace}
% Titulo em outro idioma (opcional)
% Resumo em Português
\begin{resumoumacoluna}
    \footnotesize{\begin{SingleSpace}
        \noindent
        \textbf\resumoname\\
        Conforme a ABNT NBR 6022:2018, o resumo no idioma do documento é elemento obrigatório. Constituído de uma sequência de frases concisas e objetivas e não de uma simples enumeração de tópicos, não ultrapassando 250 palavras, seguido, logo abaixo, das palavras representativas do conteúdo do trabalho, isto é, palavras-chave e/ou descritores, conforme a NBR 6028. (\ldots) As palavras-chave devem figurar logo abaixo do resumo, antecedidas da expressão Palavras-chave:, separadas entre si por ponto e finalizadas também por ponto.\\\\
        \textbf{Palavras-chave:} latex. abntex. editoração de texto.
        \vspace{\onelineskip}
    \end{SingleSpace}}
\end{resumoumacoluna}
% Resumo em Inglês
\renewcommand{\resumoname}{Abstract}
\begin{resumoumacoluna}
    \footnotesize{\begin{SingleSpace}
        \begin{otherlanguage*}{english}
            \noindent
            \textbf\resumoname\\
            Conforme a ABNT NBR 6022:2018, o resumo no idioma do documento é elemento obrigatório. Constituído de uma sequência de frases concisas e objetivas e não de uma simples enumeração de tópicos, não ultrapassando 250 palavras, seguido, logo abaixo, das palavras representativas do conteúdo do trabalho, isto é, palavras-chave e/ou descritores, conforme a NBR 6028. (\ldots) As palavras-chave devem figurar logo abaixo do resumo, antecedidas da expressão Palavras-chave:, separadas entre si por ponto e finalizadas também por ponto.\\\\
            \textbf{Keywords:} latex. abntex.
        \end{otherlanguage*}
    \end{SingleSpace}}
\end{resumoumacoluna}
% ------------------------------------------------------------------------
% ELEMENTOS TEXTUAIS
% ------------------------------------------------------------------------
\textual
\pagestyle{fancy}
% ------------------------------------------------------------------------
% INTRODUÇÃO
% ------------------------------------------------------------------------
\section{Introdução}
% ---------------------------------
% TEMA
% ---------------------------------
A inteligência artificial tem se disseminado com o tempo, e tem sido o elo que sana muitas lacunas de atividades em que máquinas e pessoas buscam harmonia. É encarada também, como um grande ganho de capital para empresas de vários ramos de atividade, inclusive na indústria. Uma das características da inteligência artificial, é o fato de otimizar uma automação já feita, em que o hardware e suas configurações já não possuem mais “inteligência” própria disponível para tanto. 

% ---------------------------------
% JUSTIFICATIVA
% ---------------------------------
Por ter variadas aplicações, a inteligência artificial em indústrias ocupa diversos setores da produção, ou manufatura de um produto. Ela está contida desde a tomada de decisão de uma prensa por exemplo, até a compilação dos relatórios gerenciais que o centro administrativo usa para tomar decisões que abrangem a planta como um todo. Um dos motivos dessa expansão, é a crescente demanda de máquinas que produzam muito, de forma customizada e de baixo custo. É uma espécie de produção programável, e inteligente. Reduzindo os gargalos de configuração e até mesmo de lentidão para produção de peças variadas com matérias primas diferentes. 

Máquinas automáticas, já conseguem trabalhar de forma mais habilidosa, do que os procedimentos manuais ou mecanizados. Modelos automáticos, já são mais eficientes e produtivos, que modelos humanos. Mas como melhorar ainda mais o rendimento de uma planta, sem necessariamente significar em investimentos com equipamentos ou procedimentos novos? Uma possível solução, seria a inteligência artificial aplicada. 

% ---------------------------------
% PROBLEMA DE PESQUISA
% ---------------------------------
Um modelo de sistema inteligente distribuído, poderia eliminar a responsabilidade do CLP de concentrar todas as informações e tomadas de decisões. Todos os equipamentos em uma planta, poderiam ser inteligentes o suficiente, para entenderem o processo de produção como um todo, e sugerir uns aos outros, formas de trabalho, modelos de atuação e comunicação durante o processo de produção. O que se é sugerido aqui, é um modelo de sistema multiagente inteligente e distribuídos, para automatizar uma planta de automação, em que a limitação de hardware já não permite mais progresso na relação custo vs. benefício de produção. 

% ---------------------------------
% OBJETIVOS
% ---------------------------------

% ---------------------------------
% PROCEDIMENTOS METODOLÓGICOS
% ---------------------------------

% ---------------------------------
% ORGANIZAÇÃO DO ARTIGO
% ---------------------------------

% ------------------------------------------------------------------------
% REFERENCIAL METODOLÓGICO
% ------------------------------------------------------------------------
\section{Referencial Metodológico}
Seção que aborda a descrição dos métodos e procedimentos utilizados.
% ---------------------------------
% CLASSIFICAÇÃO DA PESQUISA
% ---------------------------------
\subsection{Método}
A obra em questão é uma pesquisa de natureza básica, com objetivos explicativos, utilizando procedimentos fundamentados em pesquisa bibliográfica e documental. É uma \textbf{pesquisa básica} pois promove o aprofundamento e reflexão em um conhecimento já consolidado em pesquisas anteriores. Os autores esperam contribuir com visões peculiares, em relação a aplicação da inteligência artificial distribuída nas diversas atividades industriais automatizadas.

Tem objetivos \textbf{explicativos}, pois explica como a inteligência artificial distribuída pode contribuir para economia de recursos de um sistema automatizado, poupando tempo, energia, velocidade na tomada de decisão, e ainda promovendo a autonomia de processos, que sem a IA (Inteligência Artificial) se limitavam às configurações de \emph{hardware} disponíveis.

Os autores utilizaram técnicas de pequisa bibliográfica e documental pois, o \textbf{procedimento bibliográfico} proporciona a condição de comparação histórica com a literatura, registrando o avanço da tecnologia e dos métodos citados. E o \textbf{procedimento documental} confere a garantia de consulta e comparação dos artigos primários com a literatura não processada.

% ---------------------------------
% CLASSIFICAÇÃO DOS PROCEDIMENTOS
% ---------------------------------
\subsection{Procedimentos Metodológicos}
A pesquisa bibliográfica e documental possibilitou filtrar e comparar uma quantidade de publicações satisfatória, levando em conta a amplitude de tempo de evolução da tecnologia. Começando pelas primeiras aplicações industriais da IAD (Inteligência Artificial Distribuída), na seleção de rotas ótimas para a locomoção, até os sistemas mais complexos de tomada de decisão na indústria.

% ------------------------------------------------------------------------
% REFERENCIAL TEÓRICO
% ------------------------------------------------------------------------
\section{Referencial Teórico}
Seção que aborda o conhecimento fundamental para a compreensão desta pesquisa.
% ---------------------------------
% CONCEITO 01
% ---------------------------------
\subsection{Inteligência Artificial Distribuída (IAD)}
A IAD, tem uma aplicação vasta em diversas áreas, para resolver problemas da computação que envolve a tomada de decisão, economia de recursos e otimização de processos. Shih et al. (1995), a IAD é aplicada para subdividir processos que já existem, visando melhorar a utilização de uma infraestrutura pronta, para facilitar o planejamento de processos de montagem em lote. Os problemas de busca de objetivo e de busca de métodos para atingir o alvo são conhecidos e pode-se utilizar a IAD, para tornar mais eficiente as tomadas de decisões para a solução desses problemas.

% ------------------------------------------------------------------------
% RESULTADOS E DISCUSSÃO
% ------------------------------------------------------------------------
\section{Resultados e Discussão}
Esta seção tem por objetivo expor os testes realizados de maneira sistemática, com seus resultados individuais, abrindo questionamentos a respeito do conhecimento estudado. Ao final desta seção, a conclusão apresenta uma síntese do produto gerado pelos experimentos.

% ---------------------------------
% DISCUSSÃO 01
% ---------------------------------
\subsection{Vantagens da IAD na Indústria}
Objetivo do teste
Descrição do teste (condição de execução)
Resultado do teste (empregar gráficos e tabelas, se possível)
Interpretação do resultado (discutir o resultado relacionando com a literatura)
Conclusão parcial do Teste 1

% ---------------------------------
% DISCUSSÃO 02
% ---------------------------------
\subsection{Desvantagens da IAD na Indústria}
Objetivo do teste
Descrição do teste (condição de execução)
Resultado do teste (empregar gráficos e tabelas, se possível)
Interpretação do resultado (discutir o resultado relacionando com a literatura)
Conclusão parcial do Teste 2

% ---------------------------------
% CONCLUSÃO DAS DISCUSSÕES
% ---------------------------------
\subsection{Conclusão}
Síntese das conclusões parciais (Teste 1, Teste 2, ... Teste n)
Generalização dos resultados dos testes

% Finaliza a parte no bookmark do PDF, para que se inicie o bookmark na raiz
\bookmarksetup{startatroot}
% ------------------------------------------------------------------------
% CONCLUSÃO
% ------------------------------------------------------------------------
\section{Conclusão}
%\lipsum[31]
% ------------------------------------------------------------------------
% ELEMENTOS PÓS-TEXTUAIS
% ------------------------------------------------------------------------
\postextual
% ------------------------------------------------------------------------
% REFERẼNCIAS
% ------------------------------------------------------------------------
\bibliography{references}
% ------------------------------------------------------------------------
% APÊNDICES
% ------------------------------------------------------------------------
\begin{apendicesenv}
%\chapter{Cras non urna sed feugiat cum sociis natoque penatibus et magnis dis parturient montes nascetur ridiculus mus}
%\lipsum[31]
\end{apendicesenv}
% ------------------------------------------------------------------------
% ANEXOS
% ------------------------------------------------------------------------
\begin{anexosenv}
\vspace{\onelineskip}
%\chapter{Cras non urna sed feugiat cum sociis natoque penatibus et magnis dis parturient montes nascetur ridiculus mus}
%\lipsum[31]
\end{anexosenv}
% ------------------------------------------------------------------------
% AGRADECIMENTOS
% ------------------------------------------------------------------------
%\section*{Agradecimentos}
%\lipsum[31]
% ------------------------------------------------------------------------
% FINAL DO DOCUMENTO
% ------------------------------------------------------------------------
\end{document}